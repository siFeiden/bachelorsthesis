\begin{center}
\begin{minipage}[t]{0.8\textwidth}
	\section*{Abstract}
We present a semantics for a probabilistic programming language with program variables that can have values in the integers.
The semantics is based on formal power series and is an extension to previous work of Scherbaum~\cite{clara:pgf}.
Scherbaum's semantics does not allow program variables to have negative valuations.
We define a semantic rule for each of the operations in the programming language.
Loops are the most intricate operations.
To ease finding a loop's semantics, we introduce a novel concept to obtain a statement about the semantics of a loop.
It uses binary relations to either find the actual loop semantics or prove an overapproximation of the loop semantics.
An element of such a relation is a pair of input and output of a loop.
Later on, we prove that our semantics is equivalent in expectation to an already established one.
For the comparison, we chose the weakest preexpectation semantics introduced by McIver and Morgan.
We prove that the execution of a program in either semantics gives the same expected value for an arbitrary property expressed as a function over the program variables.
\end{minipage}
\end{center}
