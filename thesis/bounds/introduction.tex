Probabilistic programs are programs that do not always produce the same result for one set of inputs.
This is achieved by the usage of probabilistic operations, which are embedded into the programming language itself.
%Contrary to randomised programs, probabilistic programs do not rely on an external source of randomness, but use this inherent mechanism to make random decisions.
Applications are in testing, where error rates can be modelled easily using probabilistic programs.
They can also be used to solve problems that are very hard to solve or are not solvable at all using classical programming paradigms \cite{motwani:randomized}. \\
Because of its non-deterministic nature, code written by using a probabilistic programming language is not easily verified just by reading and understanding.
To obtain insights about properties such as correctness and termination probability, a formal method is necessary.
It should allow to derive statements about the aforementioned properties.
Unlike program code, which describes how something is calculated, the formal method must keep track of what happens while a program is executed.
What happens, or the meaning of the process, is called the program's semantics.
One idea to define a semantics is to keep track of all possible variable valuations, called the program states. \\
Probabilistic programs are not deterministic.
Consequently, program states only have a probability of appearing as result of the whole computation.
To capture all program states with their probabilities, \emph{probability generating functions} (PGFs) can be used.
This thesis is built on work of Scherbaum \cite{clara:pgf}, which uses \emph{formal power series} as PGFs.
These PGFs lack the possibility of having variables with negative values.
Thus, even simple programs that use subtraction cannot be represented by one of those PGFs.
In this work, we will extend Scherbaum's model, so that variables can also have negative values.
Later, we will show that the newly derived semantics is equivalent to an already established one which was introduced by McIver and Morgan.~\cite{mciver:abstraction_refinement}.
