\subsection*{Introduction}
In~\cite{arbab:coinduction}, Arbab et al.\ introduced the concept of \emph{coinduction}.
Coinduction can be used to reason about sequences of numbers using relations called \emph{bisimulations}.
Inspired by their ideas, we will introduce two methods that use relations of PGFs to reason about loop semantics.

\begin{definition}[Weak Bisimulation]
  Let $B$ be a Boolean condition and $P$ a program.
  A \emph{weak bisimulation} for $B$ and $P$ is a binary relation $R \subseteq D \times D$ such that
  for all $G$, $K \in D$, \\
  if $\pair{K, G} \in R$, then
  \begin{enumerate}[i)]
    \item[] Let $G' = \wh(G)$
    \item $\resN{G'} \leqD K$	\label{bisim:weak:prop1}
    \item $\pair{K - \resN{G'}, G'} \in R$	\label{bisim:weak:prop2}
  \end{enumerate}
\end{definition}

The definition for a weak bisimulation is an inductive one.
Assume the pair $(K, G)$ of PGFs is an element of a bisimulation $R$.
Using the property \ref{bisim:weak:prop2}, we can find another pair of PGFs that is also an element of $R$.
Note that the first and second pair are not necessarily distinct.
The entries $K$ and $G$ correspond to output and input of a loop.
$K$ is a candidate for the semantics of a loop, $G$ is the loop's input.
Recall the loop semantics:
\[ \smtx{\while{B}{P}}(G) = \sum_{i=0}^{\infty}  \resN{\wh^i(G)} \]
To get a statement about the loop semantics, we can repeatedly apply \ref{bisim:weak:prop2} to get new pairs of PGFs.
In ii), the second entry of the new pair is $\wh(G)$.
As we have seen before, $\wh$ applies the loop body $P$ once to its input.
Hence, every application of ii) applies the loop body $P$ once to $G$.
This means that applying ii) is similar to unrolling the loop once.
In the first entry, $\resN{G'} = \resN{\wh(G)}$ is subtracted from $K$ in an application of ii).
The subtractions cumulate with every application of ii) so that iterating ii) ad infinitum subtracts the semantics of the loop from $K$.
Unfortunately, a weak bisimulation can only prove that $K$ is an overapproximation of the loop semantics.

\begin{theorem}[label=theo:bisim:overapprox]
	Let $B$ be a boolean condition, $P$ a program and $R$ a basic bisimulation. \\
	Then \[ \pair{K, G} \in R \land G \models B
		\implies \smtx{\while{B}{P}}(G) \leqD K \]
	\begin{proof}
		See Appendix~\ref{proof:bisim:overapprox}.
	\end{proof}
\end{theorem}

Here, $G$ satisfies $B$ if the restriction $\resP{G}$ is the same as $G$:
\[ G \models B \iff \resP{G} = G \iff \resN{G} = 0 \]
In other words, every program state for which the corresponding coefficient is non-zero must satisfy the condition $B$.
This part of the precondition restricts the loops we can handle to those where the input PGF satisfies the loop condition.
In practice, many loops appearing have that property.
For example, in all loops in the previous examples, their input PGF satisfies the loop condition. \\
In a weak bisimulation, $K$ is an overapproximation for the loop semantics.
The reason is that $\resN{\wh(G)}$ in property \ref{bisim:weak:prop2} may be 0.
For example, $R_{0} = \{ (1, 0) \}$ is a valid weak bisimulation.
Applying ii) gives again $(1, 0)$.
There is no loop $P_W$ such that $\smtx{P_W}(0) = 1$ yet for all loops it holds that $\smtx{\while{B}{P}}(0) \leqD 1$.
In order to prove that a pair $(K, G)$ in a bisimulation is in fact the output and input of a loop, we need to strengthen the definition of a weak bisimulation by another property.

\begin{definition}[Strong Bisimulation]
	Let $B$ be a boolean condition and $P$ a program.
	An \emph{advanced bisimulation} for $B$ and $P$ is a binary relation $R \subseteq D \times D$
	that is a basic bisimulation with one additional property.
	For all $G, K \in D$ and a constant $\varepsilon \in (0, 1)$, if $\pair{K, G} \in R$, then
	\begin{enumerate}[i)]
		\item[] Let $G' = \wh(G)$
		\item $\resN{G'} \leqD K$	\label{bisim:strong:prop1}
		\item $\pair{K - \resN{G'}, G'} \in R$	\label{bisim:strong:prop2}
		\item $\left| \resN{G'}\right| \geq \varepsilon \cdot \l| K \r|$\label{bisim:strong:prop3}
	\end{enumerate}
\end{definition}

The strong bisimulation introduces a constant $\varepsilon$ which is the same all over $R$.
This means that we cannot choose $\varepsilon$ after every application of \ref{bisim:strong:prop2}.
$\varepsilon$ must be in the interval $(0, 1)$ so that the first and third properties can never contradict:
\ref{bisim:strong:prop1} implies $|\resN{G'}| \leq |K|$.
With $\varepsilon$ chosen greater than 1, this would be a contradiction to \ref{bisim:strong:prop3}. \\
Of course, a strong bisimulation is also a weak bisimulation because it satisfies properties \ref{bisim:weak:prop1} and \ref{bisim:weak:prop2}.
The third property iii) prevents the problem we had with weak bisimulations.
It guarantees that $K$ is strictly decreased in every application of ii).
In consequence, a pair $(K, G)$ in a strong bisimulation is a pair of output and input of a loop.

\begin{theorem}[label=theo:bisim:loopsmtx]
	Let $B$ be a boolean condition, $P$ a program and $R$ a strong bisimulation for $B$ and $P$ with constant $\varepsilon$. Then
	\[ \pair{K, G} \in R \land G \models B \implies K = \smtx{\while{B}{P}}(G) \]
	\begin{proof}
		See Appendix~\ref{proof:bisim:loopsmtx}.
	\end{proof}
\end{theorem}


\subsection*{Case study}
\begin{figure}[t]
	\begin{lstlisting}[label=prg:canonical3, caption=Canonical example]
X := 0
F := 0
while (F = 0) {
  {X := X + 1}[0.5]{F := 1}
}
	\end{lstlisting}
	\caption{A probabilistic program. \label{fig:canonical}}
\end{figure}
As an example, we will find a bisimulation for the program in Figure~\ref{fig:canonical}.
Define $B = ( F = 0 )$ as the loop condition and $P = \{X := X + 1\}[0.5]\{F := 1\}$ as the loop body.
The input of the loop is $G = 1$.
The loop's output is $K := \hf \cdot \sum_{i=0}^\infty \half^i X^{i} F^1$, as we have seen in Section~\ref{ssec:canonical_examples}.
We now want to apply the previous theorem to the loop.
To do so, we define a relation $R$ such that $\pair{K, G}$ is an element of $R$ and prove that $R$ is a strong bisimulation.
\[
	R = \{ (K, G) \} \cup \l\{ \pair{ \hf \cdot \sum_{i=n}^\infty \half^i X^i F^1 ,
	  	\half^n X^n F^0 + \half^n X^{n-1} F^1 }
		\, \middle|\, n \in \N_{>0} \r\}
\]
Apparently, $\pair{K, G}$ is an element of $R$, so the pair of output and input of the loop are in $R$.
It remains to prove that $R$ is a strong bisimulation.
\begin{proof}
	We have to show the properties~\ref{bisim:strong:prop1},~\ref{bisim:strong:prop2} and~\ref{bisim:strong:prop3} for every element of $R$.
	We do this by showing them once for $\pair{K, G}$ and then for every $n > 0$.
	During the proof, we will also have to find a value for $\varepsilon$.
	\begin{itemize}
		\item Property i) for $(K, G)$:
			\begin{align*}
				\resN{ \wh(G) } & = \resN{ \hf X^1 F^0 + \hf X^0 F^1} = \hf X^0 F^1 \\
				& \leqD \hf X^0 F^1 + \frac{1}{4} X^1 F^1 + \frac{1}{8} X^2 F^1 + \ldots \\
				&= K
			\end{align*}
		\item Property ii) for $(K, G)$:
			\begin{align*}
				 & \pair{K - \resN{ \wh(G) }, \wh(G)} \\
				=& \pair{K - \hf X^0 F^1, \hf X^1 F^0 + \hf X^0 F^1} \\
				=& \pair{ \hf \cdot \sum_{i=1}^{\infty} \half^i X^i F^1,
					\hf X^1 F^0 + \hf X^0 F^1} \\
				& \in R \text{ for } n = 1
			\end{align*}
		\item Property iii) for $(K, G)$:
			\begin{align*}
				& \l| \resN{ \wh(G) } \r| = \l| \hf X^0 F^1 \r| = \hf \\
				\text{and } & |K| = \l| \hf \cdot \sum_{i=0}^\infty \half^i X^{i} F^1 \r|
					= \hf \cdot \sum_{i=0}^\infty \half^i = 1 \\
				\implies & \l| \resN{ \wh(G) } \r| = \hf \geq \varepsilon \cdot 1 = |K|
					\text{ for } \varepsilon = \hf
			\end{align*}
	\end{itemize}
	In the proof for property iii), we chose $\varepsilon = \hf$.
	We will have to use the same value for the remaining cases or find a new value that fits all cases.
	\begin{itemize}
		\item Property i) for any $n > 0$:
			\begin{align*}
				& \resN{ \wh\l( \half^n X^n F^0 + \half^n X^{n-1} F^1 \r) } \\
				=&  \resN{ \half^{n+1} X^{n+1} F^0 + \half^{n+1} X^n F^1 } \\
				=& \half^{n+1} X^n F^1 \\
				\leqD& \half^{n+1} X^n F^1 + \half^{n+2} X^{n+1} F^1 + \ldots \\
				=& \hf \cdot \sum_{i=n}^{\infty} \half^i X^i F^1
			\end{align*}
		
		\item Property ii) for any $n > 0$:
			First of all, let
			\begin{align*}
				G_{n+1} &:= \wh\l( \half^n X^n F^0 + \half^n X^{n-1} F^1 \r) \\
				&= \half^{n+1} X^{n+1} F^0 + \half^{n+1} X^n F^1
			\end{align*}
			$G_{n+1}$ appears in both entries of the tuple in property ii):
			\begin{align*}
				 & \pair{ \hf \cdot \sum_{i=n}^{\infty} \half^i X^i F^1
				  	- \resN{ G_{n+1} } , G_{n+1} } \\
				=& \pair{ \hf \cdot \sum_{i=n}^{\infty} \half^i X^i F^1
				 	- \half^{n+1} X^n F^1 , G_{n+1} } \\
				=& \pair{ \hf \cdot \sum_{i=n+1}^{\infty} \half^i X^i F^1,
					\half^{n+1} X^{n+1} F^0 + \half^{n+1} X^n F^1 } \\
				 & \in R \text{ for } n + 1
			\end{align*}
		
		\item Property iii) for any $n > 0$:
			For the first entries of elements in $R$, the following holds:
			\begin{align*}
				& \l| \hf \cdot \sum_{i=n}^{\infty} \half^i X^i F^1 \r| \\
				=& \hf \cdot \sum_{i=n}^{\infty} \half^i \\
				=& \hf \cdot \l( \sum_{i=0}^{\infty} \half^i
					- \sum_{i=0}^{n-1} \half^i \r) \\
				=& \half^n
			\end{align*}
			For the second entries, we derive the following:
			\begin{align*}
				& \l| \resN{ \wh\l( \half^n X^n F^0 + \half^n X^{n-1} F^1 \r) } \r| \\
				=& \l| \half^{n+1} X^n F^1 \r| \\
				=& \half^{n+1}
			\end{align*}
			We see that for $\varepsilon = \frac{1}{2}$ the property is satisfied.
			Note that we do not have to change the value of $\varepsilon$ that we chose earlier.
			\begin{align*}
				& \l| \resN{\wh(\wh^n(G))} \r| \\
				=& \half^{n+1} \geq \varepsilon \half^n \\
				=& \l| \sum_{i=n}^\infty \resN{\wh^i(G)} \r|		\qedhere
			\end{align*}
	\end{itemize}
\end{proof}
We have shown that $R$ is a strong bisimulation for $B$ and $P$.
Since $G \models B$, we can apply Theorem \ref{theo:bisim:overapprox} and Theorem~\ref{theo:bisim:loopsmtx} to the loop from above.
The first thing we get is:
\[ \smtx{\while{B}{P}}(G) \leqD K \]
and secondly, the stronger version:
\[ \smtx{\while{B}{P}}(G) = K \]
