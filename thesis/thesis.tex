\documentclass[12pt]{article}

\usepackage[a4paper,scale=0.8]{geometry}
\usepackage[english]{babel} % needed for English texts
\usepackage[utf8]{inputenc} % unicode encoding
\usepackage{amsmath,amssymb,amsfonts,amsthm}
\usepackage{stmaryrd}
\usepackage{listings}
\usepackage{bussproofs}
\usepackage{hyperref}  % automagically makes clickable references, table of contents
\usepackage{thmtools}
%\usepackage{enumerate} % enumerate replaced by enumitem
\usepackage[shortlabels]{enumitem}
%\usepackage[a4,frame,center,noinfo]{crop}	% frame around pages
\usepackage{setspace}
\usepackage{pdfpages}


\newtheorem{definition}{Definition}[section] % defines the definition environment
\newtheorem{theorem}{Theorem}[section] % defines the theorem environment
\newtheorem{lemma}{Lemma}[section]
%\newtheorem{proof}{Proof}[section] % defines the proof environment

% Sets and set operation
\newcommand{\R}{\mathbb{R}} % the reals
\newcommand{\Z}{\mathbb{Z}} % the integers
\newcommand{\N}{\mathbb{N}} % the natural numbers
\renewcommand{\S}{\mathbb{S}} % state space
\newcommand{\leqD}{\sqsubseteq_{D}} % cpo
\newcommand{\leqL}{\sqsubseteq_{L}} % cpo
%\newcommand{\pot}[1]{\mathcal{P}(#1)} % power set

\newcommand{\E}{\mathbb{E}}

% Programs
\newcommand{\smtx}[1]{\left\llbracket #1 \right\rrbracket} % semantic brackets
\newcommand{\res}[2]{\left\langle #1 \right\rangle_{#2}} % restriction
\newcommand{\resP}[1]{\res{#1}{B}} % restriction to B
\newcommand{\resN}[1]{\res{#1}{\lnot B}} % restriction to not B
\newcommand{\nop}{\textit{skip}} % skip statement
\newcommand{\abort}{\textit{abort}} % abort statement
\newcommand{\ifp}[3]{\textit{if}\,(#1) \, \{ \, #2 \, \} \, \textit{else} \, \{ \, #3 \, \} } % if statement
\newcommand{\while}[2]{\textit{while}\,(#1) \{#2\}} % while loop

\newcommand{\wh}{\operatorname{wh}_{B, P}}
\newcommand{\hh}{\operatorname{H}_{B, P}}
\newcommand{\ww}{\operatorname{ww}_f}
\renewcommand{\wp}[1]{\operatorname{wp}\l( #1 \r)}

% Convenience
\newcommand{\qu}{\frac{1}{4}} % 1 / 4
\newcommand{\half}{\l( \frac{1}{2} \r)} % (1 / 2)
\newcommand{\hf}{\frac{1}{2}} % 1 / 2
\newcommand{\IH}[1]{\overset{IH}{#1}} % for induction hypothesis
\newcommand{\pair}[1]{\left( #1 \right)} % round parentheses
\renewcommand{\l}{\left} % short left
\renewcommand{\r}{\right} % short right
\newcommand{\infsum}{\sum\limits_{i = 0}^{\infty}} % sum from 0 to inf over variable i
\newcommand{\multisum}[1]{\sum\limits_{s \in \N^k} #1 X_1^{s_1} \cdots X_k^{s_k}}
\newcommand{\Xok}{X_1^{s_1} \cdots X_k^{s_k}} % variables X_1 to X_k
\newcommand{\explain}[1]{\hspace*{2em} \l(\text{\small #1}\r)} % explanation in proof, indented line with smaller text
\newcommand{\where}{, \text{ where }}

% Display
\renewcommand{\vert}{\ | \ }

% Bussproofs spacing around horizontal line
\renewcommand{\extraVskip}{5pt}

\addtolength{\jot}{0.5em} % increase align line spacing

\newcommand{\HRule}{\rule{\linewidth}{0.5mm}}	% horizontal lines for title

% math symbol for general cartesian product, copied from kpfonts.sty, lines 611 & 1495
\DeclareSymbolFont{largesymbolsA}{U}{jkpexa}{m}{n}
\DeclareMathSymbol{\bigX}{\mathop}{largesymbolsA}{16}
\newcommand{\bigtimes}{\bigX} % redundant declaration to make texstudio recognize \bigtimes


% (re)define square subsets
\DeclareFontFamily{U}{mathb}{\hyphenchar\font45}
\DeclareFontShape{U}{mathb}{m}{n}{
	<5> <6> <7> <8> <9> <10> gen * mathb
	<10.95> mathb10 <12> <14.4> <17.28> <20.74> <24.88> mathb12
}{} 
\DeclareSymbolFont{mathb}{U}{mathb}{m}{n}
\DeclareMathSymbol{\sqsubseteq} {3}{mathb}{"84}
\DeclareMathSymbol{\sqsupseteq} {3}{mathb}{"85}
\DeclareMathSymbol{\sqsubsetneq}{3}{mathb}{"88}
\DeclareMathSymbol{\sqsupsetneq}{3}{mathb}{"89}


\lstset{language=Java} % lstlisting language
\setlist[description]{font=\normalfont\space} % normal font in description env. labels



%\title{Extending Probability Generating Function Semantics to Negative Variable Valuations}
%\date{Winter Semester 2015/16}
%\author{Simon Feiden\\Supervisor: Benjamin Kaminski}


\begin{document}
	
\begin{titlepage}
	\center
	
	\vspace*{1cm}
	
	\textsc{\LARGE Rwth Aachen University}\\[2cm]
	
	
	\HRule \\[0.4cm]
	\begin{onehalfspace}
		\LARGE \bfseries Extending Probability Generating Function Semantics to Negative Variable Valuations
	\end{onehalfspace}
	\HRule \\[1.5cm]


	{\large Bachelor's Thesis} \\[0.4cm]
	{\large of Simon Feiden } \\[0.4cm]
	{\small March 29, 2016} \\[5cm]
%	{\today}\\[3cm]
	
	
	\begin{minipage}{0.7\textwidth}
		\begin{flushleft} \large
			\emph{Advisor} \\
			Benjamin Kaminski \\[0.2cm]
			\emph{First Reviewer} \\
			Prof. Dr. Ir. Joost-Pieter Katoen \\[0.2cm]
			\emph{Second Reviewer} \\
			apl. Prof. Dr. Thomas Noll
		\end{flushleft}
	\end{minipage}
	~
	\begin{minipage}{0.1\textwidth}
		\begin{flushright}
		\end{flushright}
	\end{minipage}
	
	\vfill
\end{titlepage}


\null
\thispagestyle{empty}
\newpage


\includepdf{oath.pdf}


\null
\thispagestyle{empty}
\newpage


\begin{center}
\begin{minipage}[t]{0.8\textwidth}
	\section*{Abstract}
We present a semantics for a probabilistic programming language with program variables that can have values in the integers.
The semantics is based on formal power series and is an extension to previous work of Scherbaum~\cite{clara:pgf}.
Scherbaum's semantics does not allow program variables to have negative valuations.
We define a semantic rule for each of the operations in the programming language.
Loops are the most intricate operations.
To ease finding a loop's semantics, we introduce a novel concept to obtain a statement about the semantics of a loop.
It uses binary relations to either find the actual loop semantics or prove an overapproximation of the loop semantics.
An element of such a relation is a pair of input and output of a loop.
Later on, we prove that our semantics is equivalent in expectation to an already established one.
For the comparison, we chose the weakest preexpectation semantics introduced by McIver and Morgan.
We prove that the execution of a program in either semantics gives the same expected value for an arbitrary property expressed as a function over the program variables.
\end{minipage}
\end{center}

\newpage


\tableofcontents
\newpage


\section{Introduction}
Probabilistic programs are programs that do not always produce the same result for one set of inputs.
This is achieved by the usage of probabilistic operations, which are embedded into the programming language itself.
%Contrary to randomised programs, probabilistic programs do not rely on an external source of randomness, but use this inherent mechanism to make random decisions.
Applications are in testing, where error rates can be modelled easily using probabilistic programs.
They can also be used to solve problems that are very hard to solve or are not solvable at all using classical programming paradigms \cite{motwani:randomized}. \\
Because of its non-deterministic nature, code written by using a probabilistic programming language is not easily verified just by reading and understanding.
To obtain insights about properties such as correctness and termination probability, a formal method is necessary.
It should allow to derive statements about the aforementioned properties.
Unlike program code, which describes how something is calculated, the formal method must keep track of what happens while a program is executed.
What happens, or the meaning of the process, is called the program's semantics.
One idea to define a semantics is to keep track of all possible variable valuations, called the program states. \\
Probabilistic programs are not deterministic.
Consequently, program states only have a probability of appearing as result of the whole computation.
To capture all program states with their probabilities, \emph{probability generating functions} (PGFs) can be used.
This thesis is built on work of Scherbaum \cite{clara:pgf}, which uses \emph{formal power series} as PGFs.
These PGFs lack the possibility of having variables with negative values.
Thus, even simple programs that use subtraction cannot be represented by one of those PGFs.
In this work, we will extend Scherbaum's model, so that variables can also have negative values.
Later, we will show that the newly derived semantics is equivalent to an already established one which was introduced by McIver and Morgan.~\cite{mciver:abstraction_refinement}.

\newpage


\section{Preliminaries}
\begin{frame}
	\frametitle{Formal Power Series}
	\begin{itemize}[<+->]
		\itemspacing{20pt}
		\item $ G = \alert<+(1)>{a_0} \alert<+(-1)>{X^0}
			+ \alert<.>{a_1} \alert<.(-1)>{X^1}
			+ \alert<.>{a_2} \alert<.(-1)>{X^2} + \ldots
			= \infsum \alert<.>{a_i} \alert<.(-1)>{X^i} $
		\item $ 4 X^2 + 9 X^7 $
		\item $ \infsum \half^i X^i \only<+->{ = \frac{1}{1 - \hf X} } $
		\item Multivariate FPS: $ \multisum{\mu_s} $
	\end{itemize}
\end{frame}

\begin{frame}
	\frametitle{Operations on FPS}
	
	\begin{itemize}[<+->]
		\itemspacing{10pt}
		\item Let $ G = \infsum a_i X^i $ \uncover<+->{ and $ F = \infsum b_i X^i $. }
		\item $ G + F = \infsum (a_i + b_i) X^i $
		\item $ \alpha \cdot G = \infsum (\alpha \cdot a_i) X^i $
		\item $ |G| = \infsum a_i $
		\item Analogous for multivariate FPS
	\end{itemize}
\end{frame}


\newpage


\section{Extended Semantics}
\begin{figure}[t]
	\begin{minipage}[t]{0.5\textwidth}
		\begin{lstlisting}[label=prg:canonical, caption=Canonical example]
X := 0;
F := 0;
while ( F = 0 ) {
  { X := X + 1 }[0.5]{ F := 1 }
}
		\end{lstlisting}
	\end{minipage}
	\begin{minipage}[t]{0.5\textwidth}
		\begin{lstlisting}[label=prg:canonical_neg, caption=Inverse canonical example]
X := 0;
F := 0;
while ( F = 0 ) {
  { X := X - 1 }[0.5]{ F := 1 }
}
		\end{lstlisting}
	\end{minipage}
	\caption{Two simple probabilistic programs. \label{fig:canonicals}}
\end{figure}

Having introduced a probabilistic programming language and an associated semantics, we now want to extend that semantics so that variables can also have negative valuations.
But first, let us have a look at why this is beneficial.
Even simple programs like $X := -1$ are not representable in the previous semantics because $X$ is assigned a negative value.
This might be a contrived example but for instance, every subtraction that is used in a program must be checked by hand to never have a negative result.
Consider Listings~\ref{prg:canonical} and~\ref{prg:canonical_neg} in Figure \ref{fig:canonicals}.
The first one is perfectly expressible in the semantics while the latter is not, although only a single operation was changed from $+$ to $-$. \\
We will improve on this by defining a new semantics that allows negative variable evaluations.
Additionally, the new semantics should also support closed forms.
The first is to simply extend the range of the PGFs from $\N^k$ to $\Z^k$, i.e.
\[ G = \sum_{s \in \Z^k} s X_1^{s_1} \cdots X_k^{s_k} \]
Those are still PGFs, but they no longer form a ring, because multiplication is not well-defined for every pair of PGFs with extended range.
Unfortunately, a ring is necessary for closed forms to be well-defined, so we cannot use PGFs with extended range for the new semantics.
A solution to the problem is as follows:
Given $k$ program variables, the set $\S$ of all possible variable evaluations, called \emph{state space}, is $\S = \Z^k$.
We will partition $\S$ into sets $S_1, \ldots, S_{2^k}$ such that every program variable $X_j$ always has the same sign in every state in $S_j$.
For example, if a program has 1 variable, so $k = 1$, then the state space $\S = \Z$.
The two partitions are 
\begin{align*}
	S_1 &= \N \\
	S_2 &= \Z \setminus \N = \{ -1, -2, \ldots \} =: -\N^*
\end{align*}
If a program has 2 variables, $\S = \Z^2$ and the 4 partitions are
\begin{alignat*}{5}
	S_1 = &&  \N \phantom{*}	\times &&  \N \phantom{*}	\\
	S_2 = && -\N^*				\times &&  \N \phantom{*}	\\
	S_3 = &&  \N \phantom{*}	\times && -\N^*				\\
	S_4 = && -\N^*				\times && -\N^*
\end{alignat*}
In the general case with $k > 0$ variables,
\[ S_j = \bigtimes_{i=0}^k
	\begin{cases}
		\phantom{-} \N & \text{if } b(j)_i = 0 \\
		-\N^* & \text{otherwise}
	\end{cases}
   \where 1 \leq j \leq 2^k \]
where $\bigtimes$ is the generalised cartesian product and $b(j)$ is the $j$-th element for some enumeration $b \colon \{1, \ldots, 2^k \} \to \{0, 1\}^k$.
As $b(j)$ is an element of a cartesian product, it is a tuple.
Thus, $b(j)_i$ is simply the $i$-th entry of $b(j)$.
$\{0, 1\}^k$ has $2^k$ elements, just as many as there are partitions.
The $i$th entry of a tuple in $\{0, 1\}^k$ determines if the program variable $X_i$ is negative or positive in the corresponding partition. \\
Having partitioned the state space, we can now introduce the new semantics.
It consists of tuples which have an entry for every partition of the state space.
Every entry of such a tuple is a multivariate formal power series with its range being the corresponding partition.
\begin{definition}[Semantic Tuple]
	Let $P$ be a program of $k$ variables.
	A \emph{semantic tuple} $T_G$ is an object of the form
	\[ T_G = \l( \sum_{s \in S_1} \mu_s \Xok, \ldots,
		\sum_{s \in S_{2^k}} \mu_s \Xok \r) \]
\end{definition}
Because we are using multivariate formal series which have closed forms, we can use them in the entries of the tuples.
Let us look at an example using the following program $P$.
\begin{lstlisting}
X := 0;
{ X := -9 }[0.6]{ X := 9 }
\end{lstlisting}
$P$ has one variable, so the semantic tuples have two entries.
The first entry contains the part where $X$ is positive and the second entry the part where $X$ is strictly negative.
After executing the program, $X$ has value $-9$ with probability $0.6$ and value 9 with probability $0.4$.
A corresponding semantic tuple $T_G$ in the semantics is
\[ T_G = ( 0.4 X^9, 0.6 X^{-9} ) \]
Unfortunately, the semantics becomes unwieldy when dealing with many variables, because the number of entries grows exponentially in the number of variables.
With three variables the tuples have eight entries and with four variables they have sixteen entries.
Handling sixteen or more entries is not convenient when calculating a semantics by hand.
To avoid this problem, we can fall back to using formal power series with an extended range.
\begin{theorem}
	Every PGF with extended range
	\[ G = \sum_{s \in \Z^k } \mu_s \Xok \]
	corresponds to the semantic tuple
	\[ T_G = \l( \sum_{s \in S_1} \mu_s \Xok, \ldots, \sum_{s \in S_{2^k}} \mu_s \Xok \r) \]
\end{theorem}
Every $G$ is easily decomposable into entries of a semantic tuples.
Furthermore, the extended FPSs can be added and scaled, just as the normal PGFs.
\begin{theorem}[label=theo:ext:vectorspace]
	The set of FPSs with extended range combined with addition and scalar multiplication forms a \emph{vector space}.
	\begin{proof}
		See Appendix~\ref{proof:vector_space}.
	\end{proof}
\end{theorem}
The semantics only needs these two operations to be defined, so we do not lack any expressive power from using them.
In consequence, we can use them to simplify the notation of the semantics when dealing with many variables.
Additionally, the entries of the semantics tuples are not always non-zero.
Consider the examples in Figure \ref{fig:canonicals}.
All of them contain a variable $F$ that is only ever assigned the value 0 or 1, so $F$ is always positive.
The entries where $F$ is negative are therefore 0, so they do not have to be considered when calculating a semantics.

\subsection{Programs without Loops}
Having defined objects that can model the whole state space, we can now go on to define what happens to them during the execution of a program.
First, we will treat programs without loops because their semantics is straightforward while the semantics for loops requires some more work.
We will adopt the notation that $\smtx{P}$ is the semantics function corresponding to $P$.

\subsubsection*{Basic operations}
\begin{enumerate}
	\item $\smtx{\nop}(G) = G$
	\item $\smtx{\abort}(G) = 0 = \sum\limits_{s \in \Z^k} 0 \Xok$
	\item $\smtx{X_j := e}(G) = \sum\limits_{s \in \Z^k} \mu_s X_1^{s_1} \cdots X_j^{e(s)} \cdots X_k^{s_k}$
\end{enumerate}
\subsubsection*{Composite operations}
Here, $P, P_1$ and $P_2$ are programs and $B$ is a Boolean condition.
\begin{enumerate}
	\item $\smtx{P_1; P_2}(G) = \smtx{P_2}( \smtx{P_1}(G) )$
	\item $\smtx{ \{P_1\}[p]\{P_2\} }(G)
		= p \cdot \smtx{P_1}(G) + (1-p) \cdot \smtx{P_2}(G)$ \\
		Note that both operations of the vector space are used.
	\item $\resP{\sum\limits_{s \in \Z^k} \mu_s \Xok} =
		\sum\limits_{s \in \Z^k}\begin{cases}
		\mu_s \Xok & \text{ if } s \models B \\
		0 & \text{ otherwise}
	\end{cases}$
	\item $\smtx{\ifp{B}{P_1}{P_2}}(G) = \smtx{P_1}(\resP{G}) + \smtx{P_2}(\resN{G})$
\end{enumerate}

\subsection{Domains}

We now want to define the semantics of a loop $\while{B}{P}$.
As has been done before in \cite{denHartog:verifying_prob_progs} and \cite{gretz:wp_semantics}, we will treat the loop semantics as the least fixed point of a function on an \emph{$\omega$-complete partial order}, also called \emph{domain}.
For the loop semantics, we need two $\omega$-complete partial orders, one containing all PGFs and the other all PGF-transformers. \\
An $\omega$-complete partial order~\cite{winskel:cpos} is a set equipped with a relation $\sqsubseteq$ which is a \emph{partial order} and has a completeness property:
$(P, \sqsubseteq)$ is a partial order if and only if $\sqsubseteq$ is reflexive, antisymmetric and transitive.
$(P, \sqsubseteq)$ is $\omega$-complete if every infinitely ascending chain $d_1 \sqsubseteq d_2 \sqsubseteq \ldots \in P$ has a supremum.

\begin{definition}
	The set $D_k$ is the set of all PGFs of $k$ variables whose coefficients and absolute value lie in the range $[0, 1]$.
	\[ D_k = \l\{ \sum_{s \in \Z^k} \mu_s \Xok \middle|
		\forall s \in \Z^k \colon \mu_s \in [0, 1]
		\land \sum_{s \in \Z^k} \mu_s \in [0,1] \r\} \]
	The set $L_k$ is the set of all functions from $D_k$ to $D_k$.
	\[ L_k = \l\{ D_k \to D_k \r\} \]
We equip the sets $D_k$ and $L_k$ with the following relations:
\begin{itemize}
	\item $d \sqsubseteq_{D_k} d' \iff \forall s \in \Z^k \colon \mu_s \leq \mu_s'$ \\
		A PGF $d$ is less than or equal to another PGF $d'$ iff every coefficient of $\Xok$ in $d$ is less than or equal to the corresponding coefficient in $d'$.
	\item $f \sqsubseteq_{L_k} g \iff \forall d \in D_k \colon f(d) \sqsubseteq_{D_k} g(d)$ \\
		A PGF transformer is less than or equal to another iff it is pointwise less than or equal for all elements of $D_k$.
\end{itemize}
\end{definition}
In the remaining thesis, we will leave out the index $k$ of $D_k$ and $L_k$ if $k$ is clear from context.

\begin{lemma}[label=lem:ext:cpos]
	$D_k$ and $L_k$ are $\omega$-complete partial orders.
	\begin{proof}
		See Appendix \ref{proof:cpos}.
	\end{proof}
\end{lemma}
Before we continue, we take a closer look at the sets $D$ and $L$.
Consider an element of $D$.
Elements in $D$ are restricted to having their absolute value and all coefficients between zero and one.
Each of its coefficients is the probability that the corresponding program state appears.
Neither of these probabilities can be more than one or less than zero, so these restrictions do not exclude any programs or PGFs we could use for describing actual programs. \\
Elements in $L$ are functions from PGF to PGF.
Every program transforms an initial PGF into its result.
This means that all programs must correspond to an element in $L$: $\smtx{P} \in L$.

\subsection{Programs with Loops}
\label{sec:loop_semantics}
Having introduced the two $\omega$-complete partial orders $D$ and $L$, we can go on defining the loop semantics $\smtx{P_W}$ of a loop $P_W = \while{B}{P}$.
To do so, we define two functions.
The first one is $\wh$ and it is defined as follows:
\begin{align*}
	& \wh \colon D \to D \\
	& \wh(G) = \smtx{P}(\resP{G}) \\
\end{align*}
$\wh$ applies the loop body once after restricting its argument to the loop condition.
This corresponds to the decision if the loop body should be executed once more or not.
$\wh$ chooses the parts that satisfy the condition and applies another iteration to them.
We will later see that the remaining part $(\resN{G})$ will form the semantics of the loop.
We use $\wh$ very often in the remaining thesis.
It always occurs in combination with a loop.
In this case, if not stated otherwise, the index $B$ will correspond to the loop condition and $P$ to the loop body. \\
The second function is $\hh$:
\begin{align*}
	& \hh \colon L \to L, \\
	& \hh(f)(G) = \resN{G} + f \l( \wh(G) \r)
\end{align*}
As for $\wh$, the indices $B$ and $P$ of $\hh$ also correspond to the loop condition and loop body, respectively.
Remember that $L$ contains the set of all semantics functions of all programs.
$\hh$ is a map on $L$, so it basically transforms programs.
In fact, $\hh$ acts like a single iteration of the loop $P_W$ to its argument.
The loop body $P$ is executed once unless the condition $B$ does not hold.
In this case, nothing is done.
We can see that if we apply $\hh$ to a program $P'$:
\begin{align*}
	 & \hh( \smtx{P'} )(G) \\
	 & \explain{definition of $\hh$} \\
	=& \resN{G} + \smtx{P'} \l( \wh(G) \r) \\
	 & \explain{definition of $\wh$} \\
	=& \resN{G} + \smtx{P'} \l( \smtx{P} \l( \resP{G} \r) \r) \\
	 & \explain{concatenation of programs} \\
	=& \resN{G} + \smtx{P; P'}( \resP{G} ) \\
	 & \explain{$\smtx{\nop}$ is the identity function} \\
	=& \smtx{\nop}( \resN{G} ) + \smtx{P; P'}( \resP{G} ) \\
	 & \explain{\textit{if} semantics} \\
	=& \smtx{ \ifp{B}{P; P'}{\nop} }(G)
\end{align*}
Iterating $\hh$ will mimic the behaviour of the loop $P_W$ by unrolling the loop body.
For example, applying $\hh$ twice gives
\[ \hh^2(\smtx{P'}) = \smtx{ \ifp{B}{P; \ifp{B}{P; P'}{\nop}}{\nop} } \]
which is the second unrolling of $P_W$.
If we unroll $\hh$ ad infinitum, we get a program equivalent to the loop $P_W$.
In this case, unrolling ad infinitum is equivalent to taking the least fixed point of $\hh$.
We can find the least fixed point of $\hh$ using Kleene's fixed point theorem.
\begin{theorem}[Kleene's Fixed Point Theorem~\cite{lassez:kleene_fixed_point}]
	Every Scott-continuous function $F$ over an $\omega$-complete partial order has a least fixed point which is $\sup_{n \in \N} \{ F^n(\bot) \}$.
	$\bot$ is the bottom element of the partial order.
\end{theorem}
To apply the theorem, we need to show that $\hh$ is Scott-continuous.
\begin{lemma}[label=lem:ext:scottcontinuous]
	Let $B$ be a Boolean condition and $P$ be a program.
	Then \[ \hh(f)(G) = \resN{G} + f \l( \wh(G) \r) \text{ is Scott-continuous.} \]
	\begin{proof}
		See Appendix~\ref{proof:scott_continuity}
	\end{proof}
\end{lemma}
$\hh$ satisfies the preconditions of the theorem and we get the least fixed point $F$
\[ F = \sup_{n \in \N} \l\{ \hh^n(\bot) \r\} \]
$\bot$ is the bottom element of $L$.
It is the least element according to $\leqL$.
$\leqL$ is defined pointwise using $\leqD$, so $\bot$ must always yield the least PGF, which is $0$.
Hence, \[ \bot(G) = \sum_{s \in \Z^k} 0 \Xok \]
$\hh$ applies iterations of the loop $P_W$ to its argument,
so $F$ is the semantics of the whole loop.
\begin{definition}[Loop Semantics]
	Let $B$ be a Boolean condition and $P$ a program. The semantics of the loop $\while{B}{P}$ is given by
	\[ \smtx{\while{B}{P}} = F = \sup_{n \in \N} \l\{ \hh^n(\bot) \r\} \]
\end{definition}
We now have a first characterisation of the semantics of $P_W$.
$\smtx{P_W}$ is the fixed point of unrolling the loop.
Finding suprema by hand or software can be difficult, so we will go on to find more characterisations of $\smtx{P_W}$.
A closed form of $\hh^n(\bot)$ is a first step.
\begin{lemma}[label=lem:ext:closedformhh]
	Let $B$ be a Boolean condition and $P$ a program.
	Then
	\begin{align*}
		& \hh^n(\bot)(G) = \sum_{i = 0}^{n-1} \resN{\wh^i(G)} \\
		& \text{where } \wh(G) = \smtx{P}(\resP{G}) \text{ as defined above}.
	\end{align*}
	\begin{proof}
		See Appendix~\ref{proof:closed_form_of_hh}.
	\end{proof}
\end{lemma}
We now have another form for $\hh$ and can express the loop semantics in another way.
\[ \smtx{\while{B}{P}}(G)
	= \sup_{n \in \N} \l\{ \lambda G. \sum_{i = 0}^{n-1} \resN{\wh^i(G)} \r\}(G) \]
Let us have a look at the sum inside the supremum.
Each of its summands is a PGF with non-negative coefficients.
Since PGFs are added coefficientwise, the coefficient of the whole sum for any $s \in \Z^k$ must keep increasing with $n$ going to infinity.
The supremum over all $n$ of these sums must therefore be the infinite series.
\begin{theorem}[label=theo:ext:whilesmtx]
	Let $B$ be a Boolean condition and $P$ a program.
	Then \[ \smtx{\while{B}{P}}(G) = \sum_{i = 0}^{\infty} \resN{\wh^i(G)} \]
	\begin{proof}
		See Appendix \ref{proof:series_for_while}.
	\end{proof}
\end{theorem}
The loop semantics is expressible as an infinite series of PGFs.
Series are well understood and there is a fair amount of tools to deal with them.
\newpage


\section{Case Studies}
\begin{frame}[fragile]
	\frametitle{Canonical Example}
	\begin{lstlisting}
 while ( F = 0 ) {
   { X := X - 1 }[0.5]{ F := 1 }
 }
	\end{lstlisting}
	\begin{itemize}
		\itemspacing{10pt}
		\item<2-> $ \smtx{\while{B}{P}}(G) = 
			\sum\limits_{i = 0}^{\infty} \alert<3>{ \resN{\wh^i(G)} } $
		\item<4-> $ B = (F = 0) $
		\item<5-> $ P = \{ X := X - 1 \}[0.5]\{ F := 1 \} $
		\item<6-> $ G = 1 X^0 F^0 = 1 $
	\end{itemize}
\end{frame}

\begin{frame}
	\frametitle{Canonical Example}
	\begin{itemize}[<+->]
		\itemspacing{10pt}
		\item $ \resN{\wh^1(1)} = \hf X^0 F^1 $
		\item $ \resN{\wh^2(1)} = \qu X^{-1} F^1 $
%		\item $ \ldots $
		\item $\resN{\wh^n(1)} = \half^n X^{-(n-1)} F^1 $
		\item $ \sum\limits_{i = 0}^{\infty} \resN{\wh^i(G)}
			= \alert<.>{\hf \cdot \sum\limits_{i = 0}^{\infty} \half^i X^{-i} F^1} $
		\item Closed form: $ \frac{F}{2 - X^{-1}} $
		\item Actually: $ \pair{0, \frac{F}{2 - X^{-1}}, 0, 0} $
	\end{itemize}
\end{frame}

\begin{frame}[fragile]
\frametitle{Termination Probability}
\begin{lstlisting}
while ( F = 0 ) {
  {
    { X := X + 1 }[0.5]{ diverge }
  }[0.5]{
    F := 1
  }
}
\end{lstlisting}
\begin{itemize}[<+(1)->]
	\itemspacing{10pt}
	\item $ \resN{\wh^i \l(G\r)} = \hf \cdot \l(\frac{1}{4}\r)^{i-1} X^{i-1} F^1 $
	\item $ \sum\limits_{i = 0}^{\infty} \resN{\wh^i(G)}
	= \alert{\hf \cdot \sum\limits_{i = 0}^{\infty} \l(\frac{1}{4}\r)^i X^i F^1} $
\end{itemize}
\end{frame}

\begin{frame}
	\frametitle{Termination Probability}
	\begin{itemize}[<+->]
		\itemspacing{10pt}
		
		\item Closed form: $ \frac{F}{2 - \hf X} $
		\item Termination probability using absolute value:
		\item $ \l| \hf \cdot \sum\limits_{i = 0}^{\infty} \l(\frac{1}{4}\r)^i X^i F^1 \r|
			\uncover<+->{
				= \hf \cdot \sum\limits_{i = 0}^{\infty} \l(\frac{1}{4}\r)^i 
			}
			\uncover<+->{
				=  \frac{2}{3}
			} $
		\item $ \l| \frac{F}{2 - \hf X} \r| = \frac{1}{2 - \hf} = \frac{2}{3} $
	\end{itemize}
\end{frame}

\begin{frame}[fragile]
	\frametitle{Equivalent Programs}
	\begin{columns}
		\begin{column}[t]{0.5\textwidth}
			\begin{lstlisting}[basicstyle=\scriptsize]
while ( F = 0 ) {
  { X := X + 1 }[p]{ F := 1 }
};
 
F := 0;
while ( F = 0 ) {
  { X := X - 1 }[q]{ F := 1 }
}
			\end{lstlisting}
		\end{column}
		\hspace{-20pt}
		{ \vrule width 1pt }
%		\vrule{}
		\hspace{7pt}
		\begin{column}[t]{0.5\textwidth}
			\begin{lstlisting}[basicstyle=\scriptsize]
{ F := 0 }[0.5]{ F := 1 };
if ( F = 0 ) {
  while ( F = 0 ) {
    { X := X + 1 }[p]{ F := 1 }
  }

} else {
  F := 0;
  while ( F = 0 ) {
    X := X - 1;
    { skip }[q]{ F := 1 }
  }
}
			\end{lstlisting}
		\end{column}
	\end{columns}
\end{frame}

\begin{frame}[fragile]
	\frametitle{Equivalent Programs}
	\begin{columns}
		\begin{column}[t]{0.5\textwidth}
			\begin{lstlisting}[basicstyle=\tiny]
while ( F = 0 ) {
  { X := X + 1 }[p]{ F := 1 }
};

F := 0;
while ( F = 0 ) {
  { X := X - 1 }[q]{ F := 1 }
}
			\end{lstlisting}
		\end{column}
		\begin{column}[t]{0.5\textwidth}
			\begin{lstlisting}[basicstyle=\tiny]
{ F := 0 }[0.5]{ F := 1 };
if ( F = 0 ) {
  while ( F = 0 ) {
    { X := X + 1 }[p]{ F := 1 }
  }

} else {
  F := 0;
  while ( F = 0 ) {
    X := X - 1;
    { skip }[q]{ F := 1 }
  }
}
			\end{lstlisting}
		\end{column}
	\end{columns}
	\begin{itemize}[<+->]
		\item Introduced by Kiefer et al. in 2012.
		\item Proven equivalent in distribution for $p = \frac{1}{2}$ and $q = \frac{2}{3}$.
		\item Expected value of $X$ is the same if $q = \frac{1}{2-p}$. (Gretz et al.)
		\item Today: equivalence in distribution if $q = \frac{1}{2-p}$.
	\end{itemize}
\end{frame}

\begin{frame}
	\frametitle{Equivalent Programs}
	\begin{itemize}[<+->]
		\item First program's semantics: \\[10pt]
			$ \alert<.(4)>{ \frac{(1-q) \cdot (1-p)}{1-qp} } \cdot	% factor
			\pair{
				\alert<.(2)>{ \frac{F}{1 - p X} }, \ % first entry sum
				\alert<.(4)>{ q } \cdot	% second entry factor
					\alert<.(3)>{ \frac{X^{-1} F}{1 - q X^{-1}} } % snd entry sum
				, 0, 0 % null entries
			} $
			\vspace*{20pt}
		\item Second program's semantics: \\[10pt]
			$ \pair{
				\alert<.(3)>{ \frac{1-p}{2} } \cdot	% first entry factor
					\alert<.(1)>{ \frac{F}{1 - p X} }, \ % first entry sum
				\alert<.(3)>{ \frac{1-q}{2} } \cdot	% second entry factor
					\alert<.(2)>{ \frac{X^{-1} F}{1 - q X^{-1}} } % snd entry sum
					, 0, 0 % null entries
			} $
			\vspace*{20pt}
		\item<+(3)-> Solve $ \frac{(1-q) \cdot (1-p)}{1-qp} = \frac{1-p}{2} $ and
			$ \frac{(1-q) \cdot (1-p) \cdot q}{1-qp} = \frac{1-q}{2} $.
			\vspace*{10pt}
		\item<+(3)-> Equivalent in distribution if $ q = \frac{1}{2 - p}$.
	\end{itemize}
\end{frame}

\newpage


\section{Bisimulations}
\begin{lemma}[label=lem:bisim:iteration]
	Let $B$ be a boolean condition, $P$ a program and $R$ a basic bisimulation. \\
	Then \[ \pair{K, G} \in R
		\implies \pair{K - \sum_{i = 1}^{n} \resN{\wh^i(G)}, \wh^n(G)}
		\in R \text{ for all } n \in \N \]
	\begin{proof}
		Proof by induction over $n$. \\
		\textbf{Base case} for $n = 0$:
		\begin{align*}
					 & \pair{K, G} \in R \\
			\implies & \pair{K - 0, G} \\
					 & \explain{empty sum} \\
			\implies & \pair{K - \sum_{i = 1}^{0} \resN{\wh(G)}, \wh^0(G)}
		\end{align*}
		\textbf{Induction hypothesis}:
		 \[ \pair{K, G} \in R \implies
			\pair{K - \sum_{i = 1}^{n} \resN{\wh^i(G)}, \wh^{n}(G)} \in R
			\text{ for some } n \in \N \]
		\textbf{Inductive step}: \\
		$R$ is a basic bisimulation, so we can apply the iteration step to the induction hypothesis which yields
		\begin{align*}
				 & \pair{K, G} \in R \\
				 & \explain{induction hypothesis} \\
		\implies & \pair{K - \sum_{i = 1}^{n} \resN{\wh^i(G)}, \wh^{n}(G)} \in R \\
				 & \explain{property ii) of weak bisimulations} \\
		\implies & \pair{K - \sum_{i = 1}^{n} \resN{\wh^i(G)} - \resN{\wh(\wh^n(G))}
			, \wh(\wh^{n}(G))} \in R \\
		\implies & \pair{K - \sum_{i = 1}^{n} \resN{\wh^i(G)} - \resN{\wh^{n+1}(G)}
			, \wh^{n+1}(G)} \in R \\
		\implies & \pair{K - \sum_{i = 1}^{n+1} \resN{\wh^i(G)}
			, \wh^{n+1}(G)} \in R
		\end{align*}
	\end{proof}
\end{lemma}

\begin{theorem}[continues=theo:bisim:overapprox, label=proof:bisim:overapprox]
	Let $B$ be a Boolean condition, $P$ a program and $R$ a weak bisimulation. \\
	Then \[ \pair{K, G} \in R \land G \models B
		\implies \smtx{\while{B}{P}}(G) \leqD K \]
	\begin{proof}
		\begin{align*}
					 & \pair{K, G} \in R \land G \models B \\
					 & \explain{Lemma~\ref{lem:bisim:iteration}} \\
			\implies & \pair{K - \sum_{i = 1}^{n} \resN{\wh^i(G)}, \wh^n(G)} \in R \\
					 & \land \resN{G} = 0 \\
					 & \explain{property i) of weak bisimulations} \\
			\implies & \resN{\wh\left(\wh^n(G)\right)}
				\leqD K - \sum_{i=1}^n \resN{\wh^i(G)} \\
					 & \land \resN{\wh^0(G)} = 0 \\
			\implies & \resN{\wh^0(G)} + \sum_{i=1}^n \resN{\wh^i(G)} + \resN{\wh^{n+1}(G)} \leqD K \\
			\implies & \sum_{i=0}^{n+1} \resN{\wh^i(G)} \leqD K \\
			\implies & \sup_{n \in \N} \sum_{i=0}^{n+1} \resN{\wh^i(G)}
				\leqD \sup_{n \in \N} K \\
			\implies & \smtx{\while{B}{P}}(G) \leqD K
		\end{align*}
	\end{proof}
\end{theorem}

\begin{lemma}[label=lem:bisim:iterateeps]
	Let $B$ be a Boolean condition, $P$ a program and $R$ a strong bisimulation for $B$ and $P$ with constant $\varepsilon$.
	Then,
	\[ \pair{K, G} \in R \implies \l| K \r| - \l| \sum_{i=1}^{n} \resN{\wh^i(G)} \r|
		\leq (1-\varepsilon)^n \cdot |K| \text{ for all } n \in \N \]
	\begin{proof}
		Proof by induction over $n$. \\
		\textbf{Base case} for $n = 0$:
		\[ |K| - \l| \sum_{i=1}^{0} \resN{\wh^i(G)} \r|
			= |K| - |0| \leq 1 \cdot |K| = (1-\varepsilon)^0 \cdot |K| \]
		\textbf{Induction hypothesis}:
		\[ \pair{K, G} \in R \implies \l| K \r| - \l| \sum_{i=1}^{n} \resN{\wh^i(G)} \r|
			\leq (1-\varepsilon)^n \cdot |K| \text{ for some } n \in \N \]
		\textbf{Inductive step}:	%%% REDO
		\begin{align*}
		 & \pair{K, G} \in R \\
		 & \explain{Lemma~\ref{lem:bisim:iteration}} \\
		\implies & \pair{K - \sum_{i=1}^{n+1} \resN{\wh^i(G)}, \wh^n(G)} \in R \\
		 & \explain{Property iii) of strong bisimulations} \\
		\implies & \l| \resN{ \wh\l( \wh^n(G) \r) } \r|
			 \geq \varepsilon \cdot \l| K - \sum_{i=1}^{n} \resN{\wh^i(G)} \r| \\
		 & \explain{negate, add to both sides} \\
		\implies & \l( |K| - \l| \sum_{i=1}^{n} \resN{\wh^i(G)} \r| \r)
			- \l| \resN{ \wh^{n+1}(G) } \r| \\
		 & \leq  \l( |K| - \l| \sum_{i=1}^{n} \resN{\wh^i(G)} \r| \r)
			- \varepsilon \cdot
			\l( |K| - \l| \sum_{i=1}^{n} \resN{\wh^i(G)} \r| \r) \\
		 & \explain{merge sums, factor out} \\
		\implies & |K| - \l| \sum_{i=1}^{n+1} \resN{\wh^i(G)} \r|
			\leq (1-\varepsilon) \cdot 
			\l( |K| - \l| \sum_{i=1}^{n} \resN{\wh^i(G)} \r| \r) \\
		 & \explain{induction hypothesis} \\
		\implies & |K| - \l| \sum_{i=1}^{n+1} \resN{\wh^i(G)} \r|
			\leq (1-\varepsilon) \cdot (1-\varepsilon)^n \cdot |K|
			= (1-\varepsilon)^{n+1} \cdot |K| \\
		\end{align*}
	\end{proof}
\end{lemma}

\begin{lemma}
	\label{lem:leq_and_mass_impl_equality}
	Let $G$ and $G'$ be PGFs of $k$ variables with coefficients $\mu_s$ and $\mu_s'$ for $s \in \Z^k$, respectively. Then
	$$ G \leqD G' \ \land \ |G| = |G'| \implies G = G' $$
	\begin{proof}
		Let $G$ and $G'$ be PGFs as above such that $G \leqD G'$ and $|G| = |G'|$.
		Now assume $G \ne G'$.
		Then there exists $s \in \Z^k$ with $\mu_s \ne \mu_s'$.
		Since $G \leqD G'$ it follows that $\mu_s < \mu_s'$.
		Adding all other coefficients on both sides does not break the inequation as every coefficient of $G$ is less or equal to the respective coefficient of $G'$.
		Hence,
		\[ |G| = \sum_{s \in \Z^k} \mu_s < \sum_{s \in \Z^k} \mu_s' = |G'| \]
		This is a contradiction to the precondition, so $G = G'$.
	\end{proof}
\end{lemma}

\begin{theorem}[continues=theo:bisim:loopsmtx, label=proof:bisim:loopsmtx]
	Let $B$ be a Boolean condition, $P$ a program and $R$ a strong bisimulation for $B$ and $P$ with constant $\varepsilon$. Then
	\[ \pair{K, G} \in R \land G \models B \implies K = \smtx{\while{B}{P}}(G) \]
	\begin{proof}
		\begin{align*}
		 & \pair{K, G} \in R \land G \models B \\
		 & \explain{Lemma \ref{lem:bisim:iterateeps}} \\
		\implies & |K| - \l| \sum_{i=1}^{n} \resN{\wh^i(G)} \r|
			\leq (1-\varepsilon)^n \cdot |K|
			\land \resN{wh^0(G)} = 0 \\
		\implies & |K| - \l| \sum_{i=0}^{n} \resN{\wh^i(G)} \r|
			\leq (1-\varepsilon)^n \cdot |K| \\
		\implies & \lim_{n \to \infty}
		\l( |K| - \l| \sum_{i=0}^{n} \resN{\wh^i(G)} \r| \r)
		\leq \lim_{n \to \infty} (1-\varepsilon)^n \cdot |K| \\
		\implies & \lim_{n \to \infty} \l| K \r| -
		\lim_{n \to \infty} \l| \sum_{i=0}^{n} \resN{\wh^i(G)} \r|
		\leq \lim_{n \to \infty} (1-\varepsilon)^n \cdot |K| \\
		& \text{$a_n = \sum_{i=0}^{n} \resN{\wh^i(G)}$ is monotonic and bounded by its supremum given by the cpo $D$,} \\
		& \text{so $a_n$ converges to its supremum.} \\
		\implies & \lim_{n \to \infty} \l| K \r| -
		\l| \sup_{n \in \N} \sum_{i=0}^{n} \resN{\wh^i(G)} \r|
		\leq \lim\limits_{n \to \infty} (1-\varepsilon)^n \cdot |K| \\
		\implies & \l| K \r| - \l| \smtx{\while{B}{P}}(G) \r| \leq 0 \\
		\implies & \l| K \r| \leq \l| \smtx{\while{B}{P}}(G) \r|
		\end{align*}
		\\
		Using this and Theorem \ref{theo:bisim:overapprox}, we get
		\begin{align*}
		& \pair{K, G} \in R \land G \models B \\
		& \explain{apply Theorem \ref{theo:bisim:overapprox} and the previous result} \\
		\implies & \smtx{\while{B}{P}}(G) \leqD K \land |K| \leq \l| \smtx{\while{B}{P}}(G) \r| \\
		\implies & \l| \smtx{\while{B}{P}}(G) \r| \leq |K| \land
		|K| \leq \l| \smtx{\while{B}{P}}(G) \r| \\
		\implies & |K| = \l| \smtx{\while{B}{P}}(G) \r|
		\end{align*}
		To recap, from $\pair{K, G} \in R \land G \models B$ it follows that
		\begin{itemize}
			\item $\l| \smtx{\while{B}{P}}(G) \r| = |K|$ from above, and
			\item $\smtx{\while{B}{P}}(G) \leqD K$ from Theorem \ref{theo:bisim:overapprox}.
		\end{itemize}
		These satisfy the precondition of Lemma \ref{lem:leq_and_mass_impl_equality},
		which gives us the final result
		\[ \pair{K, G} \in R \land G \models B
			\implies K = \smtx{\while{B}{P}}(G) 		\qedhere \]
	\end{proof}
\end{theorem}
\newpage


\section{Semantics Equivalence}
\begin{frame}
	\frametitle{Comparison of Semantics}
		\begin{itemize}[<+->]
			\itemspacing{7pt}
			\item Comparison of PGF and wp semantics (McIver and Morgan)
			\item Random variables: $ E = \{ \S \to \R_{\geq 0} \}$
			\item $ \operatorname{wp} \colon P \times E \to E $
			\item $ \smtx{P}(G) $ vs. $ \wp{P, f} $
			\item $ \E_\mu (f) $, where $\mu$ a measure on program states, $f$ a random variable
			\item Every PGF has a corresponding measure.
		\end{itemize}
		\uncover<+->{
			\begin{center}
				\begin{minipage}{0.9\textwidth}
					\begin{theorem}
						Given a program $P$, a measure $\mu$ and an expectation $f$, the following holds:
						\[ \E_{\smtx{P}(\mu)} (f) = \E_\mu (\wp{P, f}) \]
					\end{theorem}
				\end{minipage}
			\end{center}
		}
\end{frame}
\newpage


\section{Conclusion and Future Work}
In this bachelor thesis, we developed a PGF semantics for probabilistic programs with integer program variable valuations.
The semantics is based on semantic tuples which are tuples where each entry is a PGFs.
The number of entries grows exponentially in the number of program variables.
This means that for more than three or four program variables the tuples are hard to handle by hand because they have many entries.
However, with some knowledge about the program we can reduce the number of entries that have to be calculated to find a complete semantic description.
Another way to reduce complexity is to use PGFs with extended range.
Every of these PGFs corresponds to a semantic tuple and can therefore be used as well.
We successfully applied the new semantics to a variety of programs and could even extend the results of Gretz et al.\ concerning two equivalent programs.
We showed that the programs are not only equal in expectation but indeed equal in their semantics.
In the next step, we found a novel approach to prove the semantics of a loop.
Originally, the loop semantics is defined as the sum of all possible loop unrollings.
Now, we can also use binary relations called bisimulations.
A bisimulation is a set of pairs of PGFs.
Each pair is a pair of input and output of the loop that is treated.
There are two types of bisimulations.
Weak bisimulations allow to prove an overapproximation for the loop's semantics.
Strong bisimulations can be used to prove that a pair in the bisimulation is in fact the input and output of the loop.
Last, we showed that our PGF semantics is equivalent in expectation to the weakest preexpectation semantics introduced by McIver and Morgan.
We proved that the execution of a program in either semantics has the same expected value for an arbitrary property expressed as a function over the program variables. \\
In this thesis, we proved that two different programs are equal in their semantics.
Instead of showing equivalence or nonequivalence of two programs, one could find a measure for the similarity of two programs.
This can be done by comparing the semantics of these programs.
In our case, this would require a distance function for PGFs.
Another idea is to find more use cases and other statements that can be derived from weak and strong bisimulations.



\newpage


\section{Appendix}
\subsection{Vector Space of FPSs with Extended Range}
\label{proof:vector_space}
We want to prove that the PGFs with extended range combined with addition and scalar multiplication form a vector space.
Recall the definition of vector space.
\begin{definition}[Vector Space~\cite{brown:vector_space}]
	A vector space $V$ over a field $F$ is a non-empty set $V$ together with two functions, 
	$+ \colon V \times V \to V$ and $\cdot \colon F \times V \to V$,
	which satisfy the following properties:
	\begin{enumerate}
		\item[] Let $u, v, w \in V$ and $r, s \in F$ be arbitrarily chosen.
		\item $ v + w = w + v $
		\item $ u + (v + w) = (u + v) + w $
		\item There exists $0 \in V$ such that $ 0 + v = v  $.
		\item There exists $(-v) \in V$ such that $ v + (-v) = 0 $.
		\item $ (r \cdot s) \cdot v = r \cdot (s \cdot v) $
		\item $ r \cdot (u + v) = r \cdot u + r \cdot v $
		\item $ (r + s) \cdot v = r \cdot v + s \cdot v $
		\item $ 1 \cdot v = v $, where $1$ is the multiplicative identity of the field $F$
	\end{enumerate}
\end{definition}
For the PGFs with extended range, we have
\begin{itemize}
	\item $ V = \l\{ \sum_{s \in \Z^k} \mu_s \Xok \middle| \mu_s \in \R \r\} $
	\item $ F = \R $
\end{itemize}
For $v = \sum_{s \in \Z^k} \mu_s \Xok$ and $w = \sum_{s \in \Z^k}\mu_s' \Xok$
and $r \in \R$, the two operations are defined as follows:
\begin{itemize}
	\item $\begin{aligned}[t]
			& + \colon V \times V \to V \\
			& v + w = \sum_{s \in \Z^k} (\mu_s + \mu_s') \Xok
	\end{aligned}$
	\item $\begin{aligned}[t]
			& \cdot \colon \R \times V \to V \\
			& r \cdot v = \sum_{s \in \Z^k} (r \cdot \mu_s) \Xok
		\end{aligned}$
\end{itemize}
To prove that $(V, \cdot, +)$ is a vector space, we have to prove properties $1$ to $8$.
\begin{theorem}[continues=theo:ext:vectorspace]
	The set of FPSs with extended range combined with addition and scalar multiplication forms a \emph{vector space}.
\begin{proof}
	Let $u, v, w \in V$ with their coefficients $\sigma_s, \mu_s$ and $\tau_s$, respectively, be arbitrarily chosen.
	Additionally, let $r, s \in \R$ be arbitrarily chosen.
	\begin{enumerate}
		\item $\begin{aligned}[t]
			 & v + w \\
			=& \sum_{s \in \Z^k} (\mu_s + \tau_s) \Xok \\
			=& \sum_{s \in \Z^k} (\tau_s + \mu_s) \Xok \\
			=& w + v
		\end{aligned}$
		\item $\begin{aligned}[t]
			 & u + (v + w) \\
			=& \sum_{s \in \Z^k} (\sigma_s + (\mu_s + \tau_s)) \Xok \\
			=& \sum_{s \in \Z^k} ((\sigma_s + \mu_s) + \tau_s) \Xok \\
			=& (u + v) + w
		\end{aligned}$
		\item Let $0 = \sum_{s \in \Z^k} 0 \Xok \in V$.
		Then, \vspace{1em} \\
		$\begin{aligned}[t]
			 & 0 + v \\
			=& \sum_{s \in \Z^k} (0 + \mu_s) \Xok \\
			=& \sum_{s \in \Z^k} \mu_s \Xok \\
			=& v
		\end{aligned}$
		\item Let $-v = \sum_{s \in \Z^k} (-\mu_s) \Xok \in V$.
		Then, \vspace{1em} \\
		$\begin{aligned}[t]
			 & v + (-v) \\
			=& \sum_{s \in \Z^k} (\mu_s + (-\mu_s)) \Xok \\
			=& \sum_{s \in \Z^k} 0 \Xok \\
			=& 0
		\end{aligned}$
		\item $\begin{aligned}[t]
			 & (r \cdot s) \cdot v \\
			=& \sum_{s \in \Z^k} ((r \cdot s) \cdot \mu_s) \Xok \\
			=& \sum_{s \in \Z^k} (r \cdot (s \cdot \mu_s)) \Xok \\
			=& r \cdot \sum_{s \in \Z^k} (s \cdot \mu_s) \Xok \\
			=& r \cdot (s \cdot v)
		\end{aligned}$
		\item $\begin{aligned}[t]
			 & r \cdot (v + w) \\
			=& r \cdot \l( \sum_{s \in \Z^k} (\mu_s + \tau_s) \Xok \r) \\
			=& \sum_{s \in \Z^k} (r \cdot (\mu_s + \tau_s)) \Xok \\
			=& \sum_{s \in \Z^k} (r \cdot \mu_s + r \cdot \tau_s) \Xok \\
			=& \sum_{s \in \Z^k} (r \cdot \mu_s) \Xok
				+ \sum_{s \in \Z^k} (r \cdot \tau_s) \Xok \\
			=& r \cdot \sum_{s \in \Z^k} \mu_s \Xok
				+ r \cdot \sum_{s \in \Z^k} \tau_s \Xok \\
			=& r \cdot v + r \cdot w
		\end{aligned}$
		\item $\begin{aligned}[t]
			 & (r + s) \cdot v \\
			=& \sum_{s \in \Z^k} ((r + s) \cdot \mu_s) \Xok \\
			=& \sum_{s \in \Z^k} (r \cdot \mu_s + s \cdot \mu_s) \Xok \\
			=& \sum_{s \in \Z^k} (r \cdot \mu_s) \Xok
				+ \sum_{s \in \Z^k} (s \cdot \mu_s) \Xok \\
			=& r \cdot \sum_{s \in \Z^k} \mu_s \Xok
			+ s \cdot \sum_{s \in \Z^k} \mu_s \Xok \\
			=& r \cdot v + s \cdot v
		\end{aligned}$
		\item $\begin{aligned}[t]
			 & 1 \cdot v \\
			=& \sum_{s \in \Z^k} (1 \cdot \mu_s) \Xok \\
			=& \sum_{s \in \Z^k} \mu_s \Xok \\
			=& v
		\end{aligned}$
	\end{enumerate}
	$(V, \cdot, +)$ satisfies all 8 properties, so it is a vector space.
\end{proof}
\end{theorem}

\subsection{$D$ and $L$ Are $\boldsymbol{\omega}$-Complete Partial Orders}
\label{proof:cpos}
$D_k$ was defined as the set of all PGFs over $k$ variables with its coefficients and absolute value in $[0,1]$.
$L_k$ is the set of all functions mapping $D_k$ to $D_k$.
To show that $D_k$ and $L_k$ are $\omega$-complete partial orders~\cite{winskel:cpos}, we have to show the following:
\begin{enumerate}
	\item $D_k$ and $L_k$ are reflexive.
	\item $D_k$ and $L_k$ are transitive.
	\item $D_k$ and $L_k$ are antisymmetric.
	\item Every $\omega$-chain in $D_k$ or $L_k$ has a supremum.
\end{enumerate}
\begin{lemma}[continues=lem:ext:cpos]
	$D_k$ is an $\omega$-complete partial order for all $k \in \N \setminus \{0\}$.
	\begin{proof}
		In the following, let $r, s$ and $t$ be arbitrary elements in $D_k$ with their coefficients $\rho_i, \sigma_i$ and $\tau_i$ for $i \in \Z^k$.
		\begin{enumerate}
			\item $\forall i \in \Z^k \colon \rho_i \leq \rho_i \iff r \leqD r$
			
			\item Assume $r \leqD s$ and $s \leqD t$.
			Then \begin{align*}
				& \forall i \in \Z^k \colon \rho_i \leq \sigma_i \land
					\forall i \in \Z^k \colon \sigma_i \leq \tau_i \\
				\implies & \forall i \in \Z^k \colon \rho_i \leq \sigma_i
					\land \sigma_i \leq \tau_i \\
				\implies & \forall i \in \Z^k \colon \rho_i \leq \tau_i \\
				\implies & r \leqD t
			\end{align*}
			
			\item Assume $r \leqD s$ and $s \leqD r$.
			Then \begin{align*}
				& \forall i \in \Z^k \colon \rho_i \leq \sigma_i \land
				\forall i \in \Z^k \colon \sigma_i \leq \rho_i \\
				\implies & \forall i \in \Z^k \colon \rho_i \leq \sigma_i
					\land \sigma_i \leq \rho_i \\
				\implies & \forall i \in \Z^k \colon \sigma_i = \rho_i \\
				\implies & r = t
			\end{align*}
			
			\item Note: Because of the  various subscripts, only in this fourth item, we will denote the coefficient of a PGF $G$ corresponding to $i \in \Z^k$ with $G(i)$. \\
			Let $d_1 \leqD d_2 \leqD d_3 \leqD \ldots$ be an $\omega$-chain in $D_k$.
			The coefficients of $d_n$ are now $d_n(i)$ for $i \in \Z^k$.
			The supremum $d_{\sup}$ of the $\omega$-chain is given by
			\[ d_{\sup}
				= \sup_{n \in \N} \{d_n\}
				= \sum_{i \in \Z^k} \sup_{n \in \N}
				\l\{ d_n(i) \r\} X_1^{i_1} \cdots X_k^{i_k} \]
			The supremum $\sup_{n \in \N} \{ d_n(i) \}$ exists because $d_n(i) \leq 1$ for all $i \in \Z^k$ by definition of $D_k$.
			It remains to prove that $d_{\sup}$ is in fact the supremum of the $\omega$-chain.
			\begin{enumerate}[i)]
				\item \textit{$d_{\sup}$ is an upper bound for all $d_n$}. \\
				Assume the contrary.
				Then there is an index $n \in \N$ such that $d_{\sup} \sqsubsetneq_D d_n$.
				Then there exists an $i \in \Z^k$ such that
				$d_n(i) > d_{\sup}(i) = \sup_{n \in \N} \{ d_n(i) \}$.
				This is a contradiction to the definition of supremum, so $d_{\sup}$ is an upper bound for all $d_n$.
				
				\item \textit{There is no $d'$ with $d_n \leqD d' \sqsubsetneq_D d_{\sup}$ for all $n \in \N$}. \\
				Assume $d'$ exists.
				Then there exists $i \in \Z$ such that $d_n(i) \leq d'(i) < d_{\sup}(i) = \sup_{n \in \N} \{ d_n(i) \}$.
				This is a contradiction to the definition of supremum, so $d'$ cannot exist.
			\end{enumerate}
			By the items i) and ii) above, $d_{\sup}$ is the supremum of the $\omega$-chain.
		\end{enumerate}
		All four items above are proven, so $D_k$ is an $\omega$-complete partial order.
	\end{proof}
\end{lemma}

\begin{lemma}
	$L_k$ is an $\omega$-complete partial order for all $\N \setminus \{0\}$.
	\begin{proof}
		In the following, let $f, g$ and $h$ be arbitrary elements in $L_k$.
		\begin{enumerate}
			\item $D_k$ is reflexive, so $\forall d \in D_k \colon f(d) \leqD f(d)
			\iff f \leqL f$.
			
			\item Assume $f \leqL g$ and $g \leqL h$.
			Then \begin{align*}
				& \forall d \in D_k \colon f(d) \leqD g(d) \land
				\forall d \in D_k \colon g(d) \leqD h(d) \\
				\implies & \forall d \in D_k \colon f(d) \leqD g(d) \land g(d) \leqD h(d) \\
				& \explain{$D_k$ is transitive} \\
				\implies & \forall d \in D_k \colon f(d) \leqD h(d) \\
				& \explain{definition of $\leqL$} \\
				\implies & f \leqL h
			\end{align*}
			
			\item Assume $f \leqL g$ and $g \leqL f$.
			Then \begin{align*}
				& \forall d \in D_k \colon f(d) \leqD g(d) \land
				\forall d \in D_k \colon g(d) \leqD f(d) \\
				\implies & \forall d \in D_k \colon f(d) \leqD g(d) \land g(d) \leqD f(d) \\
				& \explain{$D_k$ is antisymmetric} \\
				\implies & \forall d \in D_k \colon f(d) = g(d) \\
				\implies & f = g
			\end{align*}
			
			\item Let $f_1 \leqL f_2 \leqL f_3 \leqL \ldots$ be an $\omega$-chain in $L_k$.
			The supremum $l_{\sup}$ of the $\omega$-chain is given by
			\[ l_{\sup}(d) = \sup_{n \in \N} \{ f_n(d) \} \]
			The supremum on the the right hand side exists because $f_1(d) \leqD f_2(d) \leqD f_3(d) \leqD \ldots$ is an $\omega$-chain in $D_k$ for every $d \in D_k$.
			It remains to prove that $l_{\sup}$ is in fact the supremum of the $\omega$-chain.
			\begin{enumerate}[i)]
				\item \textit{$f_{\sup}$ is an upper bound for all $f_n$}. \\
				Assume the contrary.
				Then there is an index $n \in \N$ such that $f_{\sup} \sqsubsetneq_L f_n$.
				Then there exists a $d \in D_k$ such that
				$f_n(d) \sqsupsetneq_L f_{\sup}(d) = \sup_{n \in \N} \{ f_n(d) \}$.
				This is a contradiction to the definition of supremum, so $f_{\sup}$ is an upper bound for all $f_n$.
				
				\item \textit{There is no $f'$ with $f_n \leqL f' \sqsubsetneq_L f_{\sup}$ for all $n \in \N$}. \\
				Assume $f'$ exists.
				Then there exists $d \in D_k$ such that $f_n(d) \leqD f'(d) \sqsubsetneq_D f_{\sup}(d) = \sup_{n \in \N} \{ f_n(d) \}$.
				This is a contradiction to the definition of supremum, so $f'$ cannot exist.
			\end{enumerate}
			By the items i) and ii) above, $f_{\sup}$ is the supremum of the $\omega$-chain.
		\end{enumerate}
		All four items above are proven, so $L_k$ is an $\omega$-complete partial order.
	\end{proof}
\end{lemma}

\subsection{Scott-continuity of $\boldsymbol{\hh}$}
\label{proof:scott_continuity}
\begin{lemma}[continues=lem:ext:scottcontinuous]
	Let $B$ be a Boolean condition and $P$ be a program.
	Then \[ \hh(f)(G) = \resN{G} + f \l( \wh(G) \r) \text{ is Scott-continuous.} \]
	\begin{proof}
		Let $f_1 \leqL f_2 \leqL f_3 \leqL \ldots$ be an $\omega$-chain in $L$.
		To show the Scott-continuity of $\hh$, it suffices to prove that
		\[ \hh\l( \sup_{n \in \N} \{ f_n \} \r) = \sup_{n \in \N} \l\{ \hh(f_i) \r\} \]
		Starting with the left hand side of the equation, we complete the proof.
		\begin{align*}
			 & \hh\l( \sup_{n \in \N} \{ f_n \} \r) \\
			=& \lambda G. \resN{G} + \l( \sup_{n \in \N} \{ f_n \} \r) ( \smtx{P}(\resP{G}) ) \\
			=& \lambda G. \resN{G} + \sup_{n \in \N} \{ f_n ( \smtx{P}(\resP{G}) ) \} \\
			=& \sup_{n \in \N} \l\{ \lambda G. \resN{G} + f_n \l( \smtx{P}(\resP{G}) \r) \r\} \\
			=& \sup_{n \in \N} \{ \hh(f_n) \}		\qedhere
		\end{align*}
	\end{proof}
\end{lemma}

\subsection{Closed Form of $\boldsymbol{\hh^n(\bot)}$}
\label{proof:closed_form_of_hh}
We want to prove a closed form for $\hh^n(\bot)(G)$.
\begin{lemma}[continues=lem:ext:closedformhh]
	\begin{proof}
		Proof by induction over $n$. \\
		\textbf{Base case} for $n = 0$: \\
		\[ \hh^0(\bot)(G) = \bot(G) = 0 = \sum_{i = 0}^{0 - 1} \resN{\wh^i(G)} \]
		Note the empty sum in the last step. \\
		\textbf{Induction hypothesis}:
		\[ \hh^n(\bot)(G) = \sum_{i = 0}^{n-1} \resN{\wh^i(G)}, \text{ for some } n \in \N \]
		\textbf{Inductive step}:
		\begin{align*}
			 & \hh^{n+1}(\bot)(G) \\
			=& \hh( \hh^n(\bot) )(G) \\
			 & \explain{definition of $\hh$} \\
			=& \resN{G} + \hh^n(\bot) ( \smtx{P}(\resP{G}) ) \\
			 & \explain{definition of $\wh$} \\
			=& \resN{G} + \hh^n(\bot) ( \wh(G) ) \\
			 & \explain{apply induction hypothesis} \\
			=& \resN{G} + \sum_{i=0}^{n-1} \resN{ \wh^i( \wh(G) )} \\
			=& \resN{G} + \sum_{i=0}^{n-1} \resN{ \wh^{i+1}(G)} \\			
			 & \explain{offset indices} \\
			=& \resN{G} + \sum_{i=1}^{n} \resN{ \wh^i(G)} \\
			 & \explain{$\wh^0$ is the identity function} \\
			 & \resN{ \wh^0(G) } + \sum_{i=1}^{n} \resN{ \wh^i(G)} \\
			=& \sum_{i=0}^{n} \resN{ \wh^i(G)}		\qedhere
		\end{align*}
	\end{proof}
\end{lemma}

\subsection{A Series As Loop Semantics}
\label{proof:series_for_while}
\begin{theorem}[continues=theo:ext:whilesmtx]
	Let $B$ be a Boolean condition and $P$ a program.
	Then \[ \smtx{\while{B}{P}}(G) = \sum_{i = 0}^{\infty} \resN{\wh^i(G)} \]
	
	\begin{proof}
		We know that
		\[ \infsum \resN{\wh^i(G)} = \lim_{n \in \N} \sum_{i=0}^{n} \resN{\wh^i(G)} \]
		Since the partial sums are monotonically increasing, we can replace $\lim$ with $\sup$.
		\begin{align*}
			 & \infsum \resN{\wh^i(G)} \\
			=& \sup_{n \in \N} \l\{ \sum_{i=0}^{n} \resN{\wh^i(G)} \r\} \\
			=& \sup_{n \in \N} \l\{ \hh^{n+1}(\bot)(G) \r\} \\
			 & \explain{$\hh^0(\bot)(G) = 0$} \\
			=& \sup_{n \in \N} \l\{ \hh^{n}(\bot)(G) \r\} \\
			 & \explain{$\hh$ is Scott-continuous} \\
			=& \sup_{n \in \N} \l\{ \hh^{n}(\bot) \r\}(G) \\
			=& \smtx{\while{B}{P}}(G)			\qedhere
			\end{align*}
	\end{proof}
\end{theorem}

\subsection{Canonical Example}
\label{proof:canonical_example}
In this proof, we will not show the exact statement for the canonical example, but a more general one.
Therefore, we slightly modify the loop body $P$ by parametrising it with a variable $p\in [0, 1]$.
\[ P_p = \{ X := X + 1 \}[p]\{ F := 1 \} \]
The statement needed for the canonical example then is the special case $p = \hf$.
\begin{lemma}
	Let $B = (F = 0)$ be a Boolean condition and $P_p$ as above with $p \in [0, 1]$.
	Then
	\[ \resN{\operatorname{wh}_{B, P_p}^i\l(1\r)}
		= (1-p) \cdot p^{i-1} X^{i-1} F^1 \where i > 0 \]
	\begin{proof}
		We will prove the equation above without the restriction to $\lnot B$ and apply it afterwards. \\
		Proof by induction over $i$. \\
		\textbf{Base Case} for $i = 1$: \\
		\[ \operatorname{wh}_{B, P_p}^1\l(1\r)
			= p^1 X^1 F^0 + (1-p)^1 X^0 F^1 \]
		\textbf{Induction Hypothesis}:
		\[ \operatorname{wh}_{B, P_p}^i\l(1\r)
			= p^i X^i F^0 + (1-p) \cdot p^{i-1} X^{i-1} F^1 \text{ for some } i > 0 \]
		\textbf{Inductive Step}:
		\begin{align*}
			 & \operatorname{wh}_{B, P_p}^{i + 1}\l(1\r) \\
			=& \operatorname{wh}_{B, P_p}\l(
				\operatorname{wh}_{B, P_p}^i\l(1 \r) \r) \\
			 & \explain{definition of $\operatorname{wh}_{B, P_p}$} \\
			=& \smtx{P}\l( \resP{ \operatorname{wh}_{B, P_p}^i\l(1\r) } \r) \\
			 & \explain{induction hypothesis} \\
			=& \smtx{P}\l( \resP{p^i X^i F^0 + (1-p) \cdot p^{i-1} X^{i-1} F^1} \r) \\
			 & \explain{$B = (F = 0)$} \\
			=& \smtx{P}\l( p^i X^i F^0 \r) \\
			 & \explain{semantics of $P$} \\
			=& p \cdot p^i X^{i+1} F^0 + (1-p) \cdot p^i X^i F^1 \\
			=& p^{i+1} X^{i+1} F^0 + (1-p) \cdot p^i X^i F^1
		\end{align*}
		The intermediate claim is proven by induction.
		The final result is now obtained by restriction to $\lnot B$:
		\[ \resN{\operatorname{wh}_{B, P_p}^i\l(1\r)}
			= \resN{p^i X^i F^0 + (1-p) \cdot p^{i-1} X^{i-1} F^1}
			= (1-p) \cdot p^{i-1} X^{i-1} F^1		\qedhere \]
	\end{proof}
\end{lemma}
Summing all $\wh^i(1)$ gives us the semantics of the more general loop:
\begin{align*}
	 & \smtx{\while{B}{P_p}}(1) \\
	=& \resN{\operatorname{wh}_{B, P_p}^0(1)}
		+ \sum_{i = 1}^{\infty}
			\resN{\operatorname{wh}_{B, P_p}^i \l(1\r)} \\
	=& 0 + \sum_{i = 1}^{\infty} (1-p) \cdot p^{i-1} X^{i-1} F^1 \\
	 & \explain{factor out, transform indices} \\
	=& (1-p) \cdot \sum_{i = 0}^{\infty} p^i X^i F^1 \\
	 & \explain{closed form of geometric series} \\
	=& \frac{(1-p) F}{1 - p X}
\end{align*}

\subsection{Example with Termination Probability $\boldsymbol{< 1}$}
\label{proof:term_prob_lt_1}
\begin{lemma}
	For the example with termination probability $< 1$,
	\[ \resN{\wh^i\l(1\r)} = \hf \cdot \l(\frac{1}{4}\r)^{i-1} X^{i-1} F^1 \where i > 0 \]
	\begin{proof}
		We will prove the equation above without the restriction to $\lnot B$ and apply it afterwards. \\
		Proof by induction over $i$. \\
		\textbf{Base Case} for $i = 1$: \\
		\[ \wh^1\l(1\r) = \qu X^1 F^0 + \hf X^0 F^1 \]
		\textbf{Induction Hypothesis}:
		\[ \wh^i\l(1\r) = \l(\qu\r)^i X^i F^0 + \hf \cdot \l(\qu\r)^{i-1} X^{i-1} F^1
			\text{ for some } i > 0 \]
		\textbf{Inductive Step}:
		\begin{align*}
			 & \wh^{i + 1}(1) \\
			=& \wh( \wh^i(1) ) \\
			 & \explain{definition of $\wh$} \\
			=& \smtx{P}\l( \resP{ \wh^i(1) } \r) \\
			 & \explain{induction hypothesis} \\
			=& \smtx{P}\l( \resP{ \l(\qu\r)^i X^i F^0 +
				\hf \cdot \l(\qu\r)^{i-1} X^{i-1} F^1 } \r) \\
			 & \explain{$B = (F = 0)$} \\
			=& \smtx{P}\l( \l(\qu\r)^i X^i F^0 \r) \\
			 & \explain{$P = \l\{ \{ X := X + 1 \}[0.5]\{ \abort \} \r\}
			 	[0.5] \{ F := 1 \}$} \\
			=& \qu \cdot \l(\qu\r)^i X^{i+1} F^0 + \hf \cdot \l(\qu\r)^i X^i F^1 \\
			=& \l(\qu\r)^{i+1} X^{i+1} F^0 + \hf \cdot \l(\qu\r)^i X^i F^1
		\end{align*}
		The intermediate claim is proven by induction.
		The final result is now obtained by restriction to $\lnot B$.
		\[ \resN{\wh^i(1)} = \resN{\l(\qu\r)^i X^i F^0 +
			\hf \cdot \l(\qu\r)^{i-1} X^{i-1} F^1}
			= \hf \cdot \l(\qu\r)^{i-1} X^{i-1} F^1			\qedhere \]
	\end{proof}
\end{lemma}

\subsection{Equivalent Programs}
\label{proof:equivalent_programs}
\subsubsection*{Consecutive Loops}
Two statements about the loop $L^-$ in the first program of Figure~\ref{fig:equivalent_progs} remain to be proved.
First, we have to show the general form of $\wh^i$ and then the simplification of the infinite sum of all these terms.
$L^-$ operates on the PGF generated from $L^+$, the first loop in the program.
After $L^+$ and before $L^-$, the variable $F$ is set to 0, which only changes $F$'s exponent in the previous PGF.
The input for $L^-$ then is the following geometric distribution $G$, as proven in~\ref{proof:canonical_example}.
\[ G = \smtx{F := 0; L^+}\pair{1, 0}
	= (1-p) \cdot \pair{\sum_{i = 0}^{\infty} p^i X^i F^0, 0} \]
\begin{lemma}
	For the loop $L^-$, the following holds:
	\begin{align*}
		 & \resN{ \wh^n\l( G \r) } \\
		=& (1-q) \cdot q^{n-1} \cdot (1-p) \cdot \pair{
			\sum_{i=0}^{\infty} p^{i+(n-1)} X^i F^1,
			\sum_{i=1}^{n-1} p^{(n-1)-i} X^{-i} F^1}
	\end{align*}
	\begin{proof}
		We will prove the equation above without the restriction to $\lnot B$ and apply it afterwards. \\
		Proof by induction over $n$. \\
		\textbf{Base Case} for $n = 1$: \\
		\begin{align*}
			 & \wh^1(G) \\
			=& q \cdot \smtx{X := X - 1}(G) + (1-q) \cdot \smtx{F := 1}(G) \\
			=& q \cdot (1-p) \cdot \pair{\sum_{i = 0}^{\infty} p^{i+1} X^i F^0,
				1 X^{-1} F^0} \\
			 & + (1-q) \cdot (1-p) \cdot \pair{\sum_{i = 0}^{\infty} p^i X^i F^1, 0} \\
			 & \explain{empty sum in second entry is zero} \\
			=& q \cdot (1-p) \cdot \pair{\sum_{i = 0}^{\infty} p^{i+1} X^i F^0,
				1 X^{-1} F^0} \\
			 & + (1-q) \cdot (1-p) \cdot \pair{\sum_{i = 0}^{\infty} p^i X^i F^1,
			 	\sum_{i=1}^{1-1} p^{(1-1)-i} X^{-i} F^q}
		\end{align*}
		\textbf{Induction Hypothesis}:
		\begin{align*}
			 & \text{For some } n > 0 \colon \\
			 & \wh^n(G) \\
			=& q^n \cdot (1-p) \cdot \pair{
			 	\sum_{i=0}^{\infty} p^{i+n} X^i F^0,
			 	\sum_{i=1}^{n} p^{n-i} X^{-i} F^0} \\
			 & + (1-q) \cdot q^{n-1} \cdot (1-p) \cdot \pair{
			 	\sum_{i=0}^{\infty} p^{i+(n-1)} X^i F^1,
			 	\sum_{i=1}^{n-1} p^{(n-1)-i} X^{-i} F^1}
		\end{align*}
		\textbf{Inductive Step}:
		One can easily verify that restricting $\wh^n(G)$ from the induction hypothesis to $B$ gives the first summand of $\wh^n(G)$, because in the second summand, $F$ always has value 1.
		\[ \resP{\wh^n(G)} = q^n \cdot (1-p) \cdot \pair{
			\sum_{i=0}^{\infty} p^{i+n} X^i F^0, \sum_{i=1}^{n} p^{n-i} X^{-i} F^0} \]
		We can use this for the inductive step.
		\begin{align*}
			 & \wh^{n+1}\pair{G} \\
			=& \wh\l( \wh^n\pair{G} \r) \\
			 & \explain{definition of $\wh$} \\
			=& \smtx{P}\l( \resP{ \wh^n\pair{G} } \r) \\
			 & \explain{induction hypothesis + equation above} \\
			=& \smtx{P}\l( q^n \cdot (1-p) \cdot \pair{
			 	\sum_{i=0}^{\infty} p^{i+n} X^i F^0,
			 	\sum_{i=1}^{n} p^{n-i} X^{-i} F^0} \r) \\
			 & \explain{apply $P$} \\
			=& q \cdot q^n \cdot (1-p) \cdot \pair{
				\sum_{i=0}^{\infty} p^{i+n+1} X^i F^0,
				p^n X^{-1} F^0 + \sum_{i=1}^{n} p^{n-i} X^{-i-1} F^0 } \\
			 & + (1-q) \cdot q^n \cdot (1-p) \cdot \pair{
			 	\sum_{i=0}^{\infty} p^{i+n} X^i F^1,
			 	 \sum_{i=1}^{n} p^{n-i} X^{-i} F^1 } \\
			 & \explain{transform sum indices in second entry} \\
			=& q^{n+1} \cdot (1-p) \cdot \pair{
				\sum_{i=0}^{\infty} p^{i+(n+1)} X^i F^0,
				p^n X^{-1} F^0 + \sum_{i=2}^{n+1} p^{(n+1)-i} X^{-i} F^0 } \\
			 & + (1-q) \cdot q^n \cdot (1-p) \cdot \pair{
				\sum_{i=0}^{\infty} p^{i+n} X^i F^1,
				\sum_{i=1}^{n} p^{n-i} X^{-i} F^1 } \\
			 & \explain{merge sum in second entry} \\
			=& q^{n+1} \cdot (1-p) \cdot \pair{
				\sum_{i=0}^{\infty} p^{i+(n+1)} X^i F^0,
				\sum_{i=1}^{n+1} p^{(n+1)-i} X^{-i} F^0 } \\
			& + (1-q) \cdot q^{(n+1)-1} \cdot (1-p) \cdot \pair{
				\sum_{i=0}^{\infty} p^{i+((n+1)-1)} X^i F^1,
				\sum_{i=1}^{n} p^{((n+1)-1)-i} X^{-i} F^1 } \\
		\end{align*}
		The induction hypothesis is proven by induction.
		The final result is now obtained by restriction to $\lnot B$.
		One can easily see that restricting $\wh^n(G)$ to $\lnot B$ leaves only the second summand, because $F$ has value 1 only in this part of the PGF.
		\begin{align*}
			 & \resN{\wh^n(G)} \\
			=& (1-q) \cdot q^{n-1} \cdot (1-p) \cdot \pair{
				\sum_{i=0}^{\infty} p^{i+(n-1)} X^i F^1,
				\sum_{i=1}^{n-1} p^{(n-1)-i} X^{-i} F^1}			\qedhere
		\end{align*}
	\end{proof}
\end{lemma}

Next we want to prove the whole semantics of $L^-$.
\begin{lemma}
	The following holds.
	\begin{align*}
		 & \smtx{L^-}\pair{(1-p) \cdot \pair{\sum_{i = 0}^{\infty} p^i X^i F^0, 0}} \\
		=& \frac{(1-q) \cdot (1-p)}{1-qp} \cdot \pair{
			\sum_{n=0}^{\infty} p^n X^n F^1,
			q \cdot \sum_{n=0}^{\infty} q^n X^{-(n+1)} F^1 }
	\end{align*}
	\begin{proof}
	\begin{align*}
		 & \smtx{L^-}\pair{(1-p) \cdot \pair{\sum_{i=0}^{\infty} p^i X^i F^0, 0}} \\
		 & \explain{loop semantics} \\
		=& \sum_{n=0}^{\infty} \resN{
			\wh^n\pair{(1-p) \cdot \pair{\sum_{i=0}^{\infty} p^i X^i F^0, 0}} } \\
		 & \explain{insert result of previous proof} \\
		=& \sum_{n=1}^{\infty} (1-q) \cdot q^{n-1} \cdot (1-p) \cdot \pair{
			\sum_{i=0}^{\infty} p^{i+(n-1)} X^i F^1,
			\sum_{i=1}^{n-1} p^{(n-1)-i} X^{-i} F^1} \\
		 & \explain{factor out} \\
		=& (1-q) \cdot (1-p) \cdot \sum_{n=1}^{\infty} q^{n-1} \cdot \pair{
			\sum_{i=0}^{\infty} p^{i+(n-1)} X^i F^1,
			\sum_{i=1}^{n-1} p^{(n-1)-i} X^{-i} F^1} \\
		 & \explain{expand sums} \\
		=& (1-q) \cdot (1-p) \cdot \Big( \\
		& \phantom{\ +} q^0 \cdot \pair{ p^0 X^0 F^1 + p^1 X^1 F^1 + \ldots,
			0 } && |\ n = 1 \\
		& + q^1 \cdot \pair{ p^1 X^0 F^1 + p^2 X^1 F^1 + \ldots,
			p^0 X^{-1} F^1 } && |\ n = 2 \\
		& + q^2 \cdot \pair{ p^2 X^0 F^1 + p^3 X^1 F^1 + \ldots,
			p^1 X^{-1} F^1 + p^0 X^{-2} F^1 } && |\ n = 3 \\
		& + q^3 \cdot \pair{ p^3 X^0 F^1 + p^4 X^1 F^1 + \ldots,
			 p^2 X^{-1} F^1 + p^1 X^{-2} F^1 + p^0 X^{-3} F^1  } && |\ n = 4 \\
		& + \ldots \Big)
	\end{align*}
	%In order to simplify the last expression, we will consider its entries separately.
	If we look at the entries in the last expression columnwise, the factors of $X^n F^1$ for $n \in \Z$ are right under each other.
	Together with the factor $q^n$ in front, they form the geometric series $\sum_{i=0}^{\infty} q^i p^i$ multiplied by a constant factor depending on $n$.
	Hence, the sum of all $X^n F^1$ simplifies to a single expression as follows:
	\begin{align*}
		& p^n \cdot \l(\sum_{i=0}^{\infty} q^i p^i\r) \cdot X^n F^1 \where n \geq 0 \\
		& q^n \cdot \l(\sum_{i=0}^{\infty} q^i p^i\r) \cdot X^{-n} F^1 \where n < 0
	\end{align*}
	Continuing the equations from above, this gives:
	\begin{align*}
		=& (1-q) \cdot (1-p) \cdot \pair{
			\sum_{n=0}^{\infty} p^n \cdot \l(\sum_{i=0}^{\infty} q^i p^i\r) \cdot X^n F^1,
			\sum_{n=1}^{\infty} q^n \cdot \l(\sum_{i=0}^{\infty} q^i p^i\r)
				\cdot X^{-n} F^1 } \\
		 & \explain{geometric series} \\
		=& (1-q) \cdot (1-p) \cdot \pair{
			\sum_{n=0}^{\infty} p^n \cdot \frac{1}{1-qp} \cdot X^n F^1,
			\sum_{n=1}^{\infty} q^n \cdot \frac{1}{1-qp} \cdot X^{-n} F^1 } \\
		 & \explain{factor out} \\
		=& \frac{(1-q) \cdot (1-p)}{1-qp} \cdot \pair{
			\sum_{n=0}^{\infty} p^n X^n F^1,
			\sum_{n=1}^{\infty} q^n X^{-n} F^1 } \\
		 & \explain{transform indices in second entry} \\
		=& \frac{(1-q) \cdot (1-p)}{1-qp} \cdot \pair{
			\sum_{n=0}^{\infty} p^n X^n F^1,
			q \cdot \sum_{n=0}^{\infty} q^n X^{-(n+1)} F^1 }
	\end{align*}
	\end{proof}
\end{lemma}


\subsubsection*{Separated Loops}
For the loop $L^-$ in the second program in Figure~\ref{fig:equivalent_progs}, we have to prove a statement about $\resN{\wh^i\pair{1, 0}}$
\begin{lemma}
	For the loop $L^-$, the following holds:
	\[ \resN{\wh^i\pair{1, 0}} = (1-q) \cdot \pair{0, q^{i-1} X^{-i} F^1} \]
	\begin{proof}
		We will prove the equation above without the restriction to $\lnot B$ and apply it afterwards. \\
		Proof by induction over $i$. \\
		\textbf{Base Case} for $i = 1$: \\
		\[ \wh^i\pair{1, 0} = \pair{0, q X^{-1} F^0} + \pair{0, (1-q) X^{-1} F^1} \]
		\textbf{Induction Hypothesis}:
		\[ \wh^i\pair{1, 0} = \pair{0, q^i X^{-i} F^0}
			+ \pair{0, (1-q) \cdot q^{i-1} X^{-i} F^1} \text{ for some } i > 0 \]
		\textbf{Inductive Step}:
		\begin{align*}
			 & \wh^{i+1}\pair{1, 0} \\
			=& \wh\l( \wh^i\pair{1, 0} \r) \\
			 & \explain{definition of $\wh$} \\
			=& \smtx{P}\l( \resP{ \wh^i\pair{1, 0} } \r) \\
			 & \explain{induction hypothesis} \\
			=& \smtx{P}\l( \resP{ \pair{0, q^i X^{-i} F^0}
				+ \pair{0, (1-q) \cdot q^{i-1} X^{-i} F^1} } \r) \\
			 & \explain{$B = (F = 0)$} \\
			=& \smtx{P}\l( \pair{0, q^i X^{-i} F^0} \r) \\
			 & \explain{$P = \l( X := X - 1; \{ \nop \}[q]\{ F := 1 \} \r)$} \\
			=& \pair{0, q^{i+1} X^{-(i-1)} F^0}
				+ \pair{0, (1-q) \cdot q^i X^{-(i+1)} F^1}
		\end{align*}
		The intermediate claim is proven by induction.
		The final result is now obtained by restriction to $\lnot B$.
		\begin{align*}
			 & \resN{\wh^i(1)} \\
			=& \resN{ \pair{0, q^i X^{-i} F^0}
				+ \pair{0, (1-q) \cdot q^{i-1} X^{-i} F^1} } \\
			=& \pair{0, (1-q) \cdot q^{i-1} X^{-i} F^1}			\qedhere
		\end{align*}
	\end{proof}
\end{lemma}

\subsection{Weak and Strong Bisimulations}
\label{proof:bisimulations}
\begin{lemma}[label=lem:bisim:iteration]
	Let $B$ be a boolean condition, $P$ a program and $R$ a basic bisimulation. \\
	Then \[ \pair{K, G} \in R
		\implies \pair{K - \sum_{i = 1}^{n} \resN{\wh^i(G)}, \wh^n(G)}
		\in R \text{ for all } n \in \N \]
	\begin{proof}
		Proof by induction over $n$. \\
		\textbf{Base case} for $n = 0$:
		\begin{align*}
					 & \pair{K, G} \in R \\
			\implies & \pair{K - 0, G} \\
					 & \explain{empty sum} \\
			\implies & \pair{K - \sum_{i = 1}^{0} \resN{\wh(G)}, \wh^0(G)}
		\end{align*}
		\textbf{Induction hypothesis}:
		 \[ \pair{K, G} \in R \implies
			\pair{K - \sum_{i = 1}^{n} \resN{\wh^i(G)}, \wh^{n}(G)} \in R
			\text{ for some } n \in \N \]
		\textbf{Inductive step}: \\
		$R$ is a basic bisimulation, so we can apply the iteration step to the induction hypothesis which yields
		\begin{align*}
				 & \pair{K, G} \in R \\
				 & \explain{induction hypothesis} \\
		\implies & \pair{K - \sum_{i = 1}^{n} \resN{\wh^i(G)}, \wh^{n}(G)} \in R \\
				 & \explain{property ii) of weak bisimulations} \\
		\implies & \pair{K - \sum_{i = 1}^{n} \resN{\wh^i(G)} - \resN{\wh(\wh^n(G))}
			, \wh(\wh^{n}(G))} \in R \\
		\implies & \pair{K - \sum_{i = 1}^{n} \resN{\wh^i(G)} - \resN{\wh^{n+1}(G)}
			, \wh^{n+1}(G)} \in R \\
		\implies & \pair{K - \sum_{i = 1}^{n+1} \resN{\wh^i(G)}
			, \wh^{n+1}(G)} \in R
		\end{align*}
	\end{proof}
\end{lemma}

\begin{theorem}[continues=theo:bisim:overapprox, label=proof:bisim:overapprox]
	Let $B$ be a Boolean condition, $P$ a program and $R$ a weak bisimulation. \\
	Then \[ \pair{K, G} \in R \land G \models B
		\implies \smtx{\while{B}{P}}(G) \leqD K \]
	\begin{proof}
		\begin{align*}
					 & \pair{K, G} \in R \land G \models B \\
					 & \explain{Lemma~\ref{lem:bisim:iteration}} \\
			\implies & \pair{K - \sum_{i = 1}^{n} \resN{\wh^i(G)}, \wh^n(G)} \in R \\
					 & \land \resN{G} = 0 \\
					 & \explain{property i) of weak bisimulations} \\
			\implies & \resN{\wh\left(\wh^n(G)\right)}
				\leqD K - \sum_{i=1}^n \resN{\wh^i(G)} \\
					 & \land \resN{\wh^0(G)} = 0 \\
			\implies & \resN{\wh^0(G)} + \sum_{i=1}^n \resN{\wh^i(G)} + \resN{\wh^{n+1}(G)} \leqD K \\
			\implies & \sum_{i=0}^{n+1} \resN{\wh^i(G)} \leqD K \\
			\implies & \sup_{n \in \N} \sum_{i=0}^{n+1} \resN{\wh^i(G)}
				\leqD \sup_{n \in \N} K \\
			\implies & \smtx{\while{B}{P}}(G) \leqD K
		\end{align*}
	\end{proof}
\end{theorem}

\begin{lemma}[label=lem:bisim:iterateeps]
	Let $B$ be a Boolean condition, $P$ a program and $R$ a strong bisimulation for $B$ and $P$ with constant $\varepsilon$.
	Then,
	\[ \pair{K, G} \in R \implies \l| K \r| - \l| \sum_{i=1}^{n} \resN{\wh^i(G)} \r|
		\leq (1-\varepsilon)^n \cdot |K| \text{ for all } n \in \N \]
	\begin{proof}
		Proof by induction over $n$. \\
		\textbf{Base case} for $n = 0$:
		\[ |K| - \l| \sum_{i=1}^{0} \resN{\wh^i(G)} \r|
			= |K| - |0| \leq 1 \cdot |K| = (1-\varepsilon)^0 \cdot |K| \]
		\textbf{Induction hypothesis}:
		\[ \pair{K, G} \in R \implies \l| K \r| - \l| \sum_{i=1}^{n} \resN{\wh^i(G)} \r|
			\leq (1-\varepsilon)^n \cdot |K| \text{ for some } n \in \N \]
		\textbf{Inductive step}:	%%% REDO
		\begin{align*}
		 & \pair{K, G} \in R \\
		 & \explain{Lemma~\ref{lem:bisim:iteration}} \\
		\implies & \pair{K - \sum_{i=1}^{n+1} \resN{\wh^i(G)}, \wh^n(G)} \in R \\
		 & \explain{Property iii) of strong bisimulations} \\
		\implies & \l| \resN{ \wh\l( \wh^n(G) \r) } \r|
			 \geq \varepsilon \cdot \l| K - \sum_{i=1}^{n} \resN{\wh^i(G)} \r| \\
		 & \explain{negate, add to both sides} \\
		\implies & \l( |K| - \l| \sum_{i=1}^{n} \resN{\wh^i(G)} \r| \r)
			- \l| \resN{ \wh^{n+1}(G) } \r| \\
		 & \leq  \l( |K| - \l| \sum_{i=1}^{n} \resN{\wh^i(G)} \r| \r)
			- \varepsilon \cdot
			\l( |K| - \l| \sum_{i=1}^{n} \resN{\wh^i(G)} \r| \r) \\
		 & \explain{merge sums, factor out} \\
		\implies & |K| - \l| \sum_{i=1}^{n+1} \resN{\wh^i(G)} \r|
			\leq (1-\varepsilon) \cdot 
			\l( |K| - \l| \sum_{i=1}^{n} \resN{\wh^i(G)} \r| \r) \\
		 & \explain{induction hypothesis} \\
		\implies & |K| - \l| \sum_{i=1}^{n+1} \resN{\wh^i(G)} \r|
			\leq (1-\varepsilon) \cdot (1-\varepsilon)^n \cdot |K|
			= (1-\varepsilon)^{n+1} \cdot |K| \\
		\end{align*}
	\end{proof}
\end{lemma}

\begin{lemma}
	\label{lem:leq_and_mass_impl_equality}
	Let $G$ and $G'$ be PGFs of $k$ variables with coefficients $\mu_s$ and $\mu_s'$ for $s \in \Z^k$, respectively. Then
	$$ G \leqD G' \ \land \ |G| = |G'| \implies G = G' $$
	\begin{proof}
		Let $G$ and $G'$ be PGFs as above such that $G \leqD G'$ and $|G| = |G'|$.
		Now assume $G \ne G'$.
		Then there exists $s \in \Z^k$ with $\mu_s \ne \mu_s'$.
		Since $G \leqD G'$ it follows that $\mu_s < \mu_s'$.
		Adding all other coefficients on both sides does not break the inequation as every coefficient of $G$ is less or equal to the respective coefficient of $G'$.
		Hence,
		\[ |G| = \sum_{s \in \Z^k} \mu_s < \sum_{s \in \Z^k} \mu_s' = |G'| \]
		This is a contradiction to the precondition, so $G = G'$.
	\end{proof}
\end{lemma}

\begin{theorem}[continues=theo:bisim:loopsmtx, label=proof:bisim:loopsmtx]
	Let $B$ be a Boolean condition, $P$ a program and $R$ a strong bisimulation for $B$ and $P$ with constant $\varepsilon$. Then
	\[ \pair{K, G} \in R \land G \models B \implies K = \smtx{\while{B}{P}}(G) \]
	\begin{proof}
		\begin{align*}
		 & \pair{K, G} \in R \land G \models B \\
		 & \explain{Lemma \ref{lem:bisim:iterateeps}} \\
		\implies & |K| - \l| \sum_{i=1}^{n} \resN{\wh^i(G)} \r|
			\leq (1-\varepsilon)^n \cdot |K|
			\land \resN{wh^0(G)} = 0 \\
		\implies & |K| - \l| \sum_{i=0}^{n} \resN{\wh^i(G)} \r|
			\leq (1-\varepsilon)^n \cdot |K| \\
		\implies & \lim_{n \to \infty}
		\l( |K| - \l| \sum_{i=0}^{n} \resN{\wh^i(G)} \r| \r)
		\leq \lim_{n \to \infty} (1-\varepsilon)^n \cdot |K| \\
		\implies & \lim_{n \to \infty} \l| K \r| -
		\lim_{n \to \infty} \l| \sum_{i=0}^{n} \resN{\wh^i(G)} \r|
		\leq \lim_{n \to \infty} (1-\varepsilon)^n \cdot |K| \\
		& \text{$a_n = \sum_{i=0}^{n} \resN{\wh^i(G)}$ is monotonic and bounded by its supremum given by the cpo $D$,} \\
		& \text{so $a_n$ converges to its supremum.} \\
		\implies & \lim_{n \to \infty} \l| K \r| -
		\l| \sup_{n \in \N} \sum_{i=0}^{n} \resN{\wh^i(G)} \r|
		\leq \lim\limits_{n \to \infty} (1-\varepsilon)^n \cdot |K| \\
		\implies & \l| K \r| - \l| \smtx{\while{B}{P}}(G) \r| \leq 0 \\
		\implies & \l| K \r| \leq \l| \smtx{\while{B}{P}}(G) \r|
		\end{align*}
		\\
		Using this and Theorem \ref{theo:bisim:overapprox}, we get
		\begin{align*}
		& \pair{K, G} \in R \land G \models B \\
		& \explain{apply Theorem \ref{theo:bisim:overapprox} and the previous result} \\
		\implies & \smtx{\while{B}{P}}(G) \leqD K \land |K| \leq \l| \smtx{\while{B}{P}}(G) \r| \\
		\implies & \l| \smtx{\while{B}{P}}(G) \r| \leq |K| \land
		|K| \leq \l| \smtx{\while{B}{P}}(G) \r| \\
		\implies & |K| = \l| \smtx{\while{B}{P}}(G) \r|
		\end{align*}
		To recap, from $\pair{K, G} \in R \land G \models B$ it follows that
		\begin{itemize}
			\item $\l| \smtx{\while{B}{P}}(G) \r| = |K|$ from above, and
			\item $\smtx{\while{B}{P}}(G) \leqD K$ from Theorem \ref{theo:bisim:overapprox}.
		\end{itemize}
		These satisfy the precondition of Lemma \ref{lem:leq_and_mass_impl_equality},
		which gives us the final result
		\[ \pair{K, G} \in R \land G \models B
			\implies K = \smtx{\while{B}{P}}(G) 		\qedhere \]
	\end{proof}
\end{theorem}
\newpage


\bibliographystyle{alpha}
\bibliography{literature}

\end{document}
